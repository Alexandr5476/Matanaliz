%\subsection{Теорема о пространстве линейных отображений, теорема Лагранжа для отображений}
\subsection*{\S\ Линейные отображения}
\vspace{7pt plus 2pt minus 2pt}
\hangindent=20pt
\colorbox[rgb]{0.973, 1, 0.588}{\sffamily\itshape Обозначения:}\begin{enumerate} % Обозначение множества линейных отображений и нормы отображения
	\item
	\urlybox{https://youtu.be/OazzLm0DJN4?si=ZlV3glhtZ5bSkYCA&t=258}{$Lin(X, Y)$}
	--- множество линейных отображений из $X$ в $Y$. $X, Y$ --- линейные пространства над \rr. \uns{($Lin(X, Y)$ является линейным пространством над \rr\ c сложением: $(A + B)(x) = A(x) + B(x)$, умножение на скаляр:
	$(\alpha A(x)) = \alpha A(x)$)}\bigskip
	\palka{\small Отображение $A \colon X \to Y$ называется линейным, если \uns{\small $\A x_1, x_2 \in X,\ \A \alpha \in \rr$} выполнено $A(\alpha x_1 + x_2) = \alpha A(x_1) + A(x_2)$}
	\item\add*{Норма линейного оператора} Пусть {\small$A \in Lin(\rmm, \rr^n)$}, тогда
	\urlybox{https://youtu.be/OazzLm0DJN4?si=JvWdSx7vQuzNPPfr&t=402}{$\|A\|$}
	\raisebox{0pt}[0pt][12pt]{${}\eqdef \sup\limits_{\substack{x \in \rmm: \\ \|x\| = 1}} \|Ax\|$} называется \urlybox{https://youtu.be/OazzLm0DJN4?si=JvWdSx7vQuzNPPfr&t=402}{нормой линейного оператора}
\end{enumerate} % Конец обозначения множества линейных отображений и нормы отображения

\begin{zam}[https://youtu.be/OazzLm0DJN4?si=PGiZPb67g0zgXEMt&t=547] % Замечания про норму линейного отображения
	\begin{enumerate}\makeatletter\renewcommand{\p@enumi}{зам.~\thezam.}\makeatother 
		\item\label{зам к опр. норм.1}
		$\|A\| \in \rr$ (конечное), т.к. из леммы \ref{оц.нормы} $\|Ax\| \< C_A\|x\| = C_A$, где \raisebox{0pt}[0pt][0pt]{$C_A = \sqrt{\sum\limits_{i, j = 1}^{n, m}a_{ij}^2}\ ,\ a_{ij}$} --- элементы матрицы $A$.\medskip
		
		\item По теореме Вейерштрасса непрерывная функция на компакте достигает своего максимального значения, и так как линейные отображения непрерывны и множество $\{\, x \mid x \in \rmm, \|x\| = 1 \,\}$ является компактом, то $\|A\| = \max_{\substack{x \in \rmm: \\ \|x\| = 1}} \|Ax\|$ \uns{(в конечномерном случае)} 
		
		\item\label{зам к опр. норм.2} $\A x \in \rmm$ выполнено $\|Ax\| \< \|A\| \cdot \|x\|$. Доказательство: для $x \ne 0_{\rmm}$ возьмём $\bar x = \sfrac{x}{\|x\|}$, тогда $\|A\bar x\| \< \|A\|$ \uns{\small(т.к. $\|A\bar x\|$ это одно из значений, по которым берётся sup в $\|A\|$)} и, домножая на $\|x\|$, получаем доказываемое неравенство \uns{\small($\|x\| \cdot \|A\bar x\| = \|x\|\cdot \|A \cdot \sfrac{x}{\|x\|}\|;\, \sfrac{1}{\|x\|}$ --- скаляр $\Rightarrow \,= \|Ax\|$)\linebreak}\vspace{-8pt}
		
		\item Если$\E C \in \rr : \A x \in \rmm$ $\|Ax\| \< C \cdot \|x\|$, то $\|A\| \< C$, потому что, поделив на $\|x\|$ первое неравенство, получаем второе \uns{\small(т.к. $\|A \cdot \sfrac{x}{\|x\|}\|$ это одно из значений, по которым берётся sup в $\|A\|$)\linebreak}
		\end{enumerate}
\end{zam} % Конец замечаний про норму линейного отображения
\vspace{-1.5ex}
\begin{lem}[https://youtu.be/OazzLm0DJN4?si=Qwpn9tVeLwf_T6DS&t=1617]{об условиях, эквивалентных непрерывности линейного оператора}
	Пусть $X, Y$ --- линейные пространства, $A \in Lin(X, Y)$, тогда эквивалентно:
	\begin{multicols}{2}\begin{enumerate}[itemindent=20pt]
		\item $A$ --- ограничено (т.е. $\|A\|$ --- конечна)
		
		\item $A$ --- непрерывно в точке $0_X$
		
		\item $A$ --- непрерывно на $X$
		
		\item $A$ --- равномерно непрерывно на $X$
	\end{enumerate}\end{multicols}
\end{lem}

\begin{prf}
	$4 \Rightarrow 3 \Rightarrow 2$ --- очевидно.
	\begin{itemize}[leftmargin=55pt]
		\item[2 $\Rightarrow$ 1:]  
		Из определения непрерывности в $0_X$: возьмём $\eps = 1$, тогда$\E \delta > 0 :{}$если $\|x\| \< \delta$, то $\|Ax\| < 1$, поэтому для $\A x : \|x\| = 1$ выполнено $\|Ax\| = \sfrac{1}{\delta} \cdot \|A (\delta x)\| \< \sfrac{1}{\delta}$, значит $\sup\|Ax\| \< \sfrac{1}{\delta}$\medskip
		
		\item[1 $\Rightarrow$ 4:] $\A \eps > 0 \E \delta = \sfrac{\eps}{\|A\|} : \A x_1, x_2$ если $\|x_2 - x_1\| < \delta$, то $\|Ax_2 - Ax_1\| \< \eps$, потому что $\|Ax_2 - Ax_1\| = \|A(x_2 - x_1)\| \< \|A\| \cdot \|x_1 - x_2\| < \|A\| \cdot \delta = \eps$
	\end{itemize}
\end{prf}
\vspace{-15pt}
\begin{teor}[https://youtu.be/OazzLm0DJN4?si=rQwhPft7RLfNzztG&t=2414]{о пространстве линейных отображений}*
	\vspace{-15pt}
	\begin{enumerate}\makeatletter\renewcommand{\p@enumi}{т.~\theteor.}\makeatother 
		\item $\|\cdot\|$ это норма в $Lin(X, Y)$
		
		\item\label{оц.норм} Если $A \in Lin(\rmm, \rr^n), B \in Lin(\rr^n, \rr^l)$, то $\|AB\|_{m,l} \< \|A\|_{m, n} \cdot \|B\|_{n,l}$
	\end{enumerate}
\end{teor}

\begin{prf}
	\begin{enumerate}
		\item Проверка аксиом нормы:
		\begin{enumerate}
		\item $\|A\| = \sup_{\substack{x \in \rmm : \\ \|x\| = 1}} \|Ax\| \biger 0$ и $\|A\| = 0 \eq A = 0_{Lin(X, Y)}$
		
		\item $\|\alpha A\| = \sup_{\substack{x \in \rmm: \\ \|x\| = 1}} \|\alpha Ax\| = |\alpha| \sup_{\substack{x \in \rmm: \\ \|x\| = 1}} \|Ax\| = |\alpha| \cdot \|A\|$
		
		\item $\|(A + B)x\| = \|Ax + Bx\| \< \|Ax\| + \|Bx\| \< \bigl(\|A\| + \|B\|\bigr) \cdot \|x\| = \|A\| + \|B\|$
		\end{enumerate}
		
	\item $\|BAx\| \< \|B\|\cdot \|Ax\| \< \|B\| \cdot \|A\|\cdot 1$ --- это выполнено для любого $x \in \rmm : \|x\| = 1$, значит выполнено и для $\sup_{\substack{x \in \rmm: \\ \|x\| = 1}}\|BAx\| = \|AB\|$  
	\end{enumerate}
\end{prf}

\begin{zam}[https://youtu.be/OazzLm0DJN4?si=gixxNia3dTgatT7R&t=3022]
	$\|A\| \stackrel{1}{=} \sup_{\substack{x \in \rmm: \\ \|x\| = 1}}\|Ax\|
	\stackrel{2}{=} \sup_{\substack{x \in \rmm: \\ \|x\| \< 1}}\|Ax\| 
	\stackrel{3}{=} \sup_{\substack{x \in \rmm: \\ \|x\| < 1}}\|Ax\| 
	\stackrel{4}{=} \sup_{\substack{x \in \rmm: \\ \|x\| \ne 0}} \frac{\|Ax\|}{\|x\|} \medskip 
	\stackrel{5}{=} \inf\{\,C\in \rr \mid \A x \in \rmm \|Ax\| \< C \cdot \|x\|\,\}$ 
	\begin{enumerate}
		\item
	\end{enumerate}
\end{zam}

\begin{teor}[https://youtu.be/OazzLm0DJN4?si=JvVhDSiKtEYa2cTH&t=3281]{Лагранжа для отображений}\label{теор.лагрнж.для отобр.}%
	Пусть отображение $F \colon E \subset \rmm \to \rr^n$ дифференцируемо на $E$ ($E$ --- открытое), $[a, b] = \{\, a + \theta (b - a)\mid \theta \in [0, 1]\,\} \subset E$, тогда$\E c \in (a, b)$, \uns{то есть $\E \theta \in (0, 1) : c = a + \theta(b - a)$} такое что
	\[\|F(b) - F(a)\| \< \|F'(c)\| \cdot \|b - a\|\] 
\end{teor}           

\begin{prf}
	Пусть $f \colon [0, 1] \to \rmm$, $f(t) = F\bigl(a + t(b - a)\bigr)$, тогда $f$ --- дифференцируема \smallskip на $[0, 1]$ и $f'(t) = F'\bigl(a + t(b-a)\bigr)(b - a)$. По теореме \ref{лагр.вект.ф.}$\E \theta \in (0, 1) : \|f(1) - f(0)\| \< \|f'(\theta)\| \cdot (1 - 0)$, {то есть\linebreak} \[\|F(b) - F(a)\| \< \|F'\bigl(a + \theta(b - a)\bigr)\cdot (b - a)\| \< \|F'(c)\|\cdot\|b - a\| \] 
\end{prf}

%\subsection{Лемма об условиях, эквивалентных непрерывности линейного оператора, теорема об обратимости линейного оператора, близкого к обратимому}

\begin{lem}[https://youtu.be/OazzLm0DJN4?si=OJoKFMQENlqrOntk&t=3846]{о норме обратного отображения}*\label{лем.усл.экв.непр.лин.оп.}
	Пусть $B \in Lin(\rmm, \rmm), \E c > 0 : \A x \in \rmm\ \ \|Bx\| \biger c\|x\|$, тогда $B$ --- обратим и $\|B^{-1}\|< \sfrac{1}{c}$
\end{lem}

\begin{prf}
	$\mathrm{Ker}\,B = \{0\}$, значит $B$ --- обратим.\\[2pt]
	Возьмём $y \in \rmm : \|y\| = 1$, тогда$\E x \in \rmm : y = Bx$, тогда $x = B^{-1}y$, и так \smallskip как $\|Bx\| \biger c\|x\|$, то $c \|B^{-1} y\| \< \|B B^{-1} y\| = \|y\|$. Это выполнено $\A y : \|y\| = 1$, поэтому $\sup_{\substack{y \in \rr^n: \\ \|y\| = 1}} \|B^{-1}y\|\< \sfrac{1}{c}$. 
\end{prf}

\begin{zam}[https://youtu.be/OazzLm0DJN4?si=xy8ay2_pL9GdhmF6&t=4135]\label{зам:оц.норм.}
	Если $A \in Lin(\rmm, \rr^n)$ --- обратим, то $\A x \in \rmm$ \quad $\|x\| = \|A^{-1}Ax\| \< \|A^{-1}\| \cdot \|Ax\|$, значит $\|Ax\| \biger \frac{1}{\|A^{-1}\|} \cdot \|x\|$
\end{zam}

\vspace{7pt plus 2pt minus 2pt}
\colorbox[rgb]{0.973, 1, 0.588}{\sffamily\itshape Обозначение:} \urlybox{https://youtu.be/OazzLm0DJN4?si=xy8ay2_pL9GdhmF6&t=4135}{$\Omega_m$} 
${}= \{\,A \in Lin(\rmm, \rmm) \mid A \text{ --- обратим} \,\} $

\begin{teor}[https://youtu.be/OazzLm0DJN4?si=h5S4HbHpa4kiqYCI&t=4443]{об обратимости линейного отображения, близкого к обратимому}\label{обр.близ.к обр.}
	\setlength\columnsep{-10pt}
	Пусть $L \in \Omega_m$, $M \in Lin(\rmm, \rmm)$ --- <<близкий к обратимому>>, то есть $\|M - L\| < \frac{1}{\|L^{-1}\|}$, тогда:
	\begin{multicols}{2}\begin{enumerate}[itemindent=20pt]
		\item $M \in \Omega_m$
		
		\item $\|M^{-1}\| \< \cfrac{1}{\|L^{-1}\|^{-1} - \|M - L\|}$
		
		\item $\|M^{-1} - L^{-1}\| \<{}$
		{\small$\cfrac{\|L^{-1}\|}{\|L^{-1}\|^{-1} - \|M - L\|}$}$\|M - L\|$
	\end{enumerate}\end{multicols}
\end{teor}

\begin{prf} Первые два пункта получаются c помощью леммы \ref{лем.усл.экв.непр.лин.оп.}: \vspace{-10pt}
	\begin{gather*}
	\|Mx\| = \|Lx - (L - M)x\| \stackrel{\parbox{.8cm}{\tiny нер-во\\тр-ка}}{\biger} \|Lx\| - \|(L - M) x\| \stackrel{\text{\ref{зам:оц.норм.}}}{\biger}\\ \biger \frac{1}{\|L^{-1}\|}\|x\| - \|L - M\| \cdot \|x\| = \left(\frac{1}{\|L^{-1}\|} - \|L - M\|\right)\cdot \|x\| 
	\end{gather*}%
	Третий пункт:\\
	$\|M^{-1} - L^{-1}\| = \|M^{-1}(L - M)L^{-1}\| \stackrel{\text{\ref{оц.норм}}}{\<} \|M^{-1}\| \cdot \|L - M\| \cdot \|L^{-1}\| \<{}$ {\small$\frac{\|L^{-1}\|}{\|L^{-1}\|^{-1} - \|M - L\|}$}$\|M - L\|$
\end{prf}

\begin{slv}[https://youtu.be/OazzLm0DJN4?si=jcJT02o6rUzSbaNd&t=5155]\add{Непрерывность вычисления обратного оператора}
	\textit{Непрерывность вычисления обратного оператора.}\\[3pt]
	Отображение $f\colon \Omega_m \to \Omega_m, \ f(L) = L^{-1}$ --- непрерывно, так как выполнено определение непрерывности: из последнего пункта теоремы \ref{обр.близ.к обр.} получаем
	\[\A \eps > 0 \E \delta = \eps \cdot \left(\|L^{-1}\|\cdot(\|L^{-1}\| + \eps)\right)^{-1} : \A M \in \Omega_m \text{ если } \|M - L\| < \delta \text{, то } \|M^{-1} - L^{-1}\| < \eps \]  
\end{slv}
