\subsection{Теорема Лагранжа, градиент, производная по вектору, по направлению, экстремальное свойство градиента}

\begin{teor}[https://www.youtube.com/live/oWtiSJdhQV8?si=c7uqK7x7CuZSChAM&t=866]{Лагранжа для векторонозначных функций}
	$f\colon [a, b] \subset \rr \to \rmm$ --- непрерывна на $[a, b]$, дифференцируема на $(a, b)$. Тогда$\E c \in (a, b):$ \[\|f(b) - f(a)\| \< \|f'(c)\|\cdot\|(b-a)\|\]
\end{teor} % Конец теоремы Лагранжа для векторнозначных функций

\begin{prf} % Доказательство теоремы Лагранжа для векторнозначных функций
	Пусть $\varphi \colon [a, b] \to \rr $,\quad$\varphi(t) = \bscal{f(b) - f(a), f(t) - f(a)}$. \smallskip Тогда $\varphi$ --- непрерывна на $[a, b]$, дифференцируема на $(a, b)$ и $\varphi(a) = 0,\  \varphi(b) = \|f(b) - f(a)\|^2$. Поэтому
	\begin{gather*} \|f(b) - f(a)\|^2 = \varphi(b) - \varphi(a) \stackrel{\parbox{2.35cm}{\tiny \uns{по обычной теореме Лагранжа$\hspace{-3.5pt}\E c \in~\!\!(a, b)$}}}{=} \varphi'(c)(b - a) \stackrel{\text{\circl{2}}}{=} \bscal{f(b) - f(a), f'(c)}(b-a) \stackrel{\parbox{1.57cm}{\tiny \uns{нер-во Коши-\\Буняковского}}}{\<}\\
		\<\|f(b) - f(a)\|\cdot \|f'(c)\| \cdot (b - a)
	\end{gather*}
	Теперь, деля на $\|f(b) - f(a)\|$ \uns{(при $f(b) = f(a)$ доказываемое неравенство очевидно)} получаем то, что нужно. 
\end{prf} % Конец доказательства теоремы Лагранжа для векторнозначных функций

\begin{opr} % Определение градиента
	Пусть функция $f\colon E \subset \rmm \to \rr$ --- дифференцируема в точке $a \in \Int E$. Тогда матрица якоби функции $f$ имеет размер $1 \times m$ (строка). Если её транспонировать и считать, что это вектор из \rmm, то определение дифференцирцемости можно записать так:
	\[f(a + h) = f(a) + \scal{f'(a), h} + o(h), \quad \text{ при } h \to 0_{\rmm} \]
	и тогда вектор $f'(a) \in \rmm$ называется \urlybox{https://www.youtube.com/live/oWtiSJdhQV8?si=0R7V2OPxS-nBZmFw&t=1658}{градиентом} функции $f$ в точке $a$, обозначается $\grad f(a)$. 
\end{opr} % Конец определения градиента

\begin{opr} % Определение производной по верктору
	\urlybox{https://www.youtube.com/live/oWtiSJdhQV8?si=rxn2tFpsv1v4kt0u&t=1943}{Производной по вектору} $h \in \rmm$ функции $f\colon E \subset \rmm \to \rr$ в точке $a$ называется
	\[\lim_{t \to 0} \frac{f(a + th) - f(a)}{t}\]
	обозначение: $\displaystyle\pfrac{f}{h}(a)$. \urlybox{https://www.youtube.com/live/oWtiSJdhQV8?si=veALxjYiQ6bewYkp&t=2054}{Напрвлением} в \rmm\ называется вектор $l \in \rmm : \|l\| = 1$. Можно\smallskip\ рассматривать производную по направлению.
\end{opr} % Конец определения производной по ветору

\begin{zam}[https://www.youtube.com/live/oWtiSJdhQV8?si=yPj-7gaxv1RXGnDk&t=2189]
	$f\colon E \subset \rmm \to \rr$ --- дифференцируема в точке $a \in \Int E$, тогда
	\[\pfrac{f}{h}(a) = \lim_{t \to 0} \frac{f(a + th) - f(a)}{t} \stackrel{\text{опр. дифф-сти}}{=}\lim_{t \to 0} \frac{\pfrac{f}{x_1}(a)\,th_1 + \dots + \pfrac{f}{x_m}(a)\,th_m + o(t)}{t}  = \scal{\grad f(a), h}\]
\end{zam}

\begin{teor}[https://www.youtube.com/live/oWtiSJdhQV8?si=nwBS8xFjs_GBMoRl&t=2418]{Экстремальное свойство градиента}
	$f\colon E \subset \rmm \to \rr$ --- дифференцируема в точке $a \in \Int E$,\quad$\grad f(a) \ne 0$, пусть {$l = \small\cfrac{\grad f(a)}{\|\grad f(a) \|}$ \uns{--- направление в \rmm}}. Тогда $l$ --- направление наискорейшего возрастания функции $f$, т.е.
	\[\A h \in \rmm \text{, у которого } \|h\| = 1\text{ выполнено}\qquad -\|\grad f(a)\| \< \pfrac{f}{h}(a) \< \|\grad f(a)\|\]
	а равенство достигается при $h = l$ (справа) и $h = -l$ (слева)
\end{teor}
\begin{prf}
	содержимое...
\end{prf}