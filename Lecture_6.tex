\subsection{Последовательность функций, поточечная и равномерная сходимсть}

\begin{opr} % Определение последовательности функций
	\urlybox{https://www.youtube.com/live/oGN0SkfpZME?si=2FLP54qNNy38b76F&t=8739}{Последовательность функций} --- это отображение из $\mathbb{N}$ в множество функций.  
\end{opr} % Конец определения последовательности функций

\begin{opr} % Определение поточечной сходимости
	Пусть $X$ --- множество, $Y$ --- метрическое пространство, $f, f_1, f_2, \dots \colon X \to Y$, последовательность отображений $f_n$
	\urlybox{https://www.youtube.com/live/oGN0SkfpZME?si=DUpMyEo3A_0XinWZ&t=8777}{сходится поточечно}
	к отображению $f$ на множестве $E \subset X$ означает, что 
	$\A x_0 \in E\ f_n(x_0) \xrightarrow[n \to \infty]{} f(x_0)$, т.е.
	\[\A x_0 \in E\ \ \A \eps > 0 \E N : \A n > N \text{ выполнено } \bmetr{f_n(x_0), f(x_0)} < \eps\]
\end{opr} % Конец опредления поточечной сходимости

\begin{opr} % Определение равномерной сходимости
	Последовательность отображений $f_n$ \urlybox{https://www.youtube.com/live/oGN0SkfpZME?si=xiWWx6bJ14dtlytn&t=9762}{сходится равномерно} к отображению $f$ на множестве $E$ если 
	$\sup\limits_{x \in E} \metr{f_n(x_0), f(x_0)} \xrightarrow[n \to \infty]{} 0$, т.е.
	\[\A \eps > 0 \E N : \A n > N\ \ \A x \in E \text{ выполнено } \bmetr{f_n(x), f(x)} < \eps\] 
	Обозначение: \begin{tikzcd}
		f_n \ar[yshift=2pt]{r}{E}
		\ar[yshift=-2pt]{r}[swap]{n \to \infty} & f
	\end{tikzcd}
\end{opr} % Конец определения равномерной сходимости

\begin{zam}[https://www.youtube.com/live/oGN0SkfpZME?si=Xt7wgIwZyTC3gAy9&t=10296] % Замечания про равномерную сходимость
	\begin{enumerate}
		\item Из равномерной сходимости следует поточечная сходимость \uns{(наоборот нет)}
		
		\item Если \begin{tikzcd}
			f_n \ar[yshift=2pt]{r}{E}
			\ar[yshift=-2pt]{r}[swap]{n \to \infty} & f
		\end{tikzcd} и $E_0 \subset E$, то \begin{tikzcd}
		f_n \ar[yshift=2pt]{r}{E_0}
		\ar[yshift=-2pt]{r}[swap]{n \to \infty} & f
		\end{tikzcd}
	\end{enumerate}
\end{zam} % Конец замечаний про равномерную сходимость

\begin{lem}[https://www.youtube.com/live/oGN0SkfpZME?si=v9AomZJHSzhlMvMm&t=10556]{или следующий пункт замечания}\label{метр.для отобр.}
	Пусть $X$ --- множество, $Y$ --- метрическое пространство,\quad 
	$\mathcal F = \{\, f \colon X \to Y \mid f\text{ --- ограничено}\,\}$ \linebreak 
	({\small$f$ --- ограничено означает, что$\E y_0 \in Y, r \in \rr : \A x \in X$ выполнено $f(x) \in \B{y_0, r}$}). 
	Тогда функция $\rho_{\mathcal F} \colon \mathcal{F} \times \mathcal{F} \to \rr$ такая, что 
	$\metr[\mathcal F]{f_1, f_2} = \sup\limits_{x \in X} \bmetr{f_1(x), f_2(x)}$ 
	является метрикой на $\mathcal F$.
\end{lem} % Конец леммы про метрику
	
\begin{prf} % Доказательство леммы про метрику
	Выполнение \smallskip первых двух аксиом метрики (\ref{опр:метр.}) следует из их выполнения в метрике на $Y$. Неравенство треугольника: при любом $x \in X$ выполнено
	\[\bmetr{f_1(x), f_2(x)} \< \bmetr{f_1(x), g(x)} + \bmetr{g(x), f_2(x)} \< \metr[\mathcal F]{f_1, g} + \metr[\mathcal F]{g, f_2} \qquad \A f_1, f_2, g \in \mathcal F\]
	Правая часть неравенства не зависит от $x$, поэтому она является верхней границей (для множества чисел $\{\,\bmetr{f_1(x), f_2(x)} \mid x \in X\,\}$), тогда она больше \smallskip либо равна точной верхней границы  $\Rightarrow \metr[\mathcal F]{f_1, f_2} \< \metr[\mathcal F]{f_1, g} + \metr[\mathcal F]{g, f_2}$
\end{prf} % Конец доказательства леммы про метрику

\subsection{Tеорема Стокса-Зайдля}

\begin{teor}[https://www.youtube.com/live/oGN0SkfpZME?si=rzyEi2Hd6nOeRw-i&t=11638]{Стокса-Зайдля о непрерывности предельной функции}\label{ст-зд}
	Отображение $f$ и последовательность отображений $f_n$ действуют $X \to Y$, где $X, Y$ --- метрические пространства. Пусть все отображения из последовательнсти непрерывны в точке $c \in X$ и\begin{tikzcd}
		f_n \ar[yshift=2pt]{r}{X}
		\ar[yshift=-2pt]{r}[swap]{n \to \infty} & f
	\end{tikzcd}\!. Тогда $f$ непрервна в точке $c$.
\end{teor} % Конец теоремы Стокса-Зайдля

\begin{prf} % Доказательство теоремы Стокса-Зайдля
	Применяя два раза неравенство треугольника к $\bmetr{f(x) - f(c)}$, получаем 
	\[\bmetr{f(x) - f(c)} \< \bmetr{f(x) - f_n(x)} + \bmetr{f_n(x) - f(c)}\< \bmetr{f(x) - f_n(x)} + \bmetr{f_n(x) - f_n(c)} + \bmetr{f_n(c) - f(c)}\]
	Из определения равномерной сходимости $f_n$ к $f$
	\uns{\small($\A \eps > 0 \E N : \A n > N\ \sup_{x \in X} \bmetr{f_n(x) - f(x)} < \eps$)}\linebreak
	 получаем, что \rselection{$\A \eps > 0$} первое и последние слагаемое в правой части неравенства $< \eps$. Из определения непрерывности $f_n$ в точке $c$ 
	 \uns{\small($\A \eps > 0 \E U(c) :{}$ если $x \in U(c)$, то $\bmetr{f_n(x) - f_n(c)} < \eps$)}
	 получаем, что \rselection{$\E U(c)$ --- окрестность точки $x$ такая, что если $x \in U(c)$, то} второе слагаемое из правой части неравенства $< \eps$. Складывая, получаем, что \rselection{$\bmetr{f(x) - f(c)} < 3 \cdot \eps$}.
	 Получилось \rselection{определение непрервности $f$ в точке $c$.} 
\end{prf} % Конец доказатлеьства теоремы Стокса-Зайдля 

\begin{slv}[https://www.youtube.com/live/oGN0SkfpZME?si=8DDuSzi86njoQaMf&t=11737]\label{непр.на X}
	Если $f_n \in C(X)$ и\begin{tikzcd}
		f_n \ar[yshift=2pt]{r}{X}
		\ar[yshift=-2pt]{r}[swap]{n \to \infty} & f
	\end{tikzcd}\!, то $f \in C(X)$. 
\end{slv} % Конец следствия про непрерывность функции, к которой сходятся равномерно непрерывные функции

\begin{zam}[https://www.youtube.com/live/oGN0SkfpZME?si=ZupIRj7L9IorPjcV&t=12462]
	\begin{enumerate}
		\item В теореме \ref{ст-зд} достаточно того, чтобы $X$ было топологическим пространством.
		
		\item В теореме \ref{ст-зд} достаточно требовать равномерную сходимость $f_n$ к $f$ только в некоторой окрестности точки $c$.
		
		\item В следствии (\ref{непр.на X}) достаточно требовать локальную равномерную сходимость, то есть $\A x \in X \E U(x) :$\begin{tikzcd}
			f_n \ar[yshift=2pt]{r}{U(x)}
			\ar[yshift=-2pt]{r}[swap]{n \to \infty} & f
		\end{tikzcd}\!. \medskip Из локальной равномерной сходимости не следует обычная. Например, $X = (0, 1)$, $f_n(x) = x^n$: 
		\begin{description} % Начало примера
		\item[\normalfont Поточечная сходимость:] $x^n \xrightarrow[n \to \infty]{} 0$ на $(0, 1)$.
		 
		\item[\normalfont Локальная равномерная сходимость:] $\sup\limits_{(\alpha, \beta)}|x^n - 0| = \beta^n \xrightarrow[n \to \infty]{} 0 \qquad \A (\alpha, \beta) \subset (0, 1),\ \beta \ne 1$
		
		\item[\normalfont Обычной равномерной сходимости нет:] $\sup\limits_{(0, 1)}|x^n - 0| = 1$
		\end{description} % Конец примера
	\end{enumerate} % Конец списка замечаний
\end{zam}
	