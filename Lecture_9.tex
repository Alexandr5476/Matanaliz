%\subsection{Следствия для рядов, признаки Вейерштрасса, критерий Больцано-Коши равномерной сходимости функционального ряда}

\begin{opr}\label{опр:равн.сх.ряда}\add*{Равномерная сходимость функционального ряда}% Определение сходимости ряда
	Пусть $u_n \colon X \to \rr$, где $X$ --- множество, тогда
	\urlybox{https://www.youtube.com/live/g4Zgeu8xe-Q?si=ZMomj6U3_OK2CNNm&t=7696}{функциональный ряд}
	{\small $\sum\limits_{n = 1}^{\infty} u_n(x)$}
	\urlybox{https://www.youtube.com/live/g4Zgeu8xe-Q?si=KXds3sSFWT4pzL1I&t=7859}{сходится равномерно}
	(\urlybox{https://www.youtube.com/live/g4Zgeu8xe-Q?si=29zAxmSsStGu0Xpn&t=7734}{поточечно}),
	если сходится равномерно (поточечно) функциональная последовательность из
	\urlybox{https://www.youtube.com/live/g4Zgeu8xe-Q?si=YcMbU0tF5Akx5h-C&t=7850}{частичных сумм}
	{\small$S_k(x) = \sum\limits_{n = 1}^{k}u_n(x)$}. Функция $S(x) = \lim\limits_{k \to \infty} S_k(x)$ называется суммой функционального ряда. То есть ряд сходится равномерно, если
	\[\A \eps > 0 \E N : \A k > N\ \ \A x \in X \text{ выполнено } |S(x) - S_k(x)| < \eps\]
	$R_n(x) = S(x) - S_k(x) ={}$ {\small$\sum\limits_{k = n + 1}^{\infty}u_n(x)$} называется
	\urlybox{https://www.youtube.com/live/g4Zgeu8xe-Q?si=zc86BXEg-dBqU_h2&t=7990}{остатком}
	функционального ряда.
\end{opr} % Конец определения сходимостм ряда

\sbox{\rav}{\begin{tikzcd}[ampersand replacement=\&]
		R_n(x) \ar[yshift=2pt]{r}{E}
		\ar[yshift=-2pt]{r}[swap]{n \to \infty} \& 0	\end{tikzcd}}

\begin{zam}[https://www.youtube.com/live/g4Zgeu8xe-Q?si=yY3d19C42v_Bhy6E&t=7907]\label{зам:равн.сх.ряда}\begin{enumerate} % Замечания к определению сходимости ряда
	\item Ряд равномерно сходится на $E$ \eq\raisebox{0pt}[15pt][0pt]{\usebox{\rav}} \uns{\small(следует прямо из определения)}
	
	\item Eсли ряд $\sum\limits_{n = 1}^{\infty} u_n(x)$ равномерно сходится на $E$, то\begin{tikzcd}
		u_n(x) \ar[yshift=2pt]{r}{E}
		\ar[yshift=-2pt]{r}[swap]{n \to \infty} & 0
	\end{tikzcd} \uns{\small(т. к. $u_k(x) = R_{k - 1}(x) - R_k(x)$)}	
\end{enumerate}\end{zam} % Конец замечаний к опредлению сходимости ряда

\begin{teor}[https://www.youtube.com/live/g4Zgeu8xe-Q?si=vIjDFMc8UzpC7diY&t=8564]{Признак Вейерштрасса равномерной сходимости функционального ряда}\label{пр.вейер.}
	Пусть $u_n\colon X \to \rr$ \uns{(X --- множество)}, $C_n \in \rr : |u_n| \< C_n\ \uns{\A n \in \mathbb N}$ и  \raisebox{0pt}[0pt]{$\sum\limits_{n = 1}^{\infty} C_n$} --- сходится. Тогда функциональный ряд \raisebox{0pt}[0pt][0pt]{$\sum\limits_{n = 1}^{\infty} u_n(x)$} --- сходится равномерно на $X$.
\end{teor} % Конец теоремы признака Вейерштрасса

\begin{prf} % Доказательство признака Вейерштрасса
	Чтобы доказать равномерную сходимость функционального ряда, можно проверить сходится ли равномерно остататок ряда к нулю (\ref{зам:равн.сх.ряда}.1)
	\[\sup_{x \in X}\left|\sum_{k = n + 1}^{\infty} u_k(x) \right| \< \sup_{x \in X} \sum_{k = n + 1}^{\infty} |u_k(x)| \< \sum_{k = n + 1}^{\infty} C_n \xrightarrow[k \to \infty]{} 0\]
\end{prf} % Конец доказательства признака Вейетштрасса

\begin{zam}[https://www.youtube.com/live/g4Zgeu8xe-Q?si=0lZiyn6N9I6rSJEM&t=9401]\label{кр.бол.-кош.для ряд.}\add*{Формулировка критерия Больцано-Коши для равномерной сходимости рядов} % Замечание: критерий Больцано-Коши
	\textit{Критерий Больцано-Коши:} ряд $\sum\limits_{n = 1}^{\infty} u_n(x)$ равномерно сходится на $E$\eq
	\[\A \eps > 0 \E N : \A n > N,\ \A k \in \mathbb N,\ \A x \in E \text{ выполнено } |u_{n + 1}(x) + u_{n + 2}(x) + \ldots + u_{n + k}(x)| < \eps \]\vspace{-7pt}
	\palka{\small Это верно, потому что это критерий Больцано-Коши сходимости функциональной последовательности (\ref{больц.-коши для посл.}), записанный для частичных сумм, а равномерная сходимость ряда это равномерная сходимость последовательности из его частичных сумм.}\\[5pt]
	Тогда ряд не сходится равномерно \eq
	\[\E \eps > 0 : \A N \E n > N, \E k \in \mathbb N, \E x \in E : \text{ выполнено } |u_{n + 1}(x) + u_{n + 2}(x) + \ldots + u_{n + k}(x)| > \eps \]
\end{zam} % Конец замечания: критерий Больцано-Коши

\begin{teor}[https://www.youtube.com/live/g4Zgeu8xe-Q?si=p-ZJAVhu-jWLePNZ&t=10612]{Стокса-Зайдля для рядов}[]*\label{ст-зд ряды}%
	\addtocounter{nteor}{-5}%
	\addcontentsline{toc}{subsection}{\textcolor[rgb]{1, 0.478, 0.612}{\arabic{nteor}}\quad Следствие для рядов (из теоремы Стокса-Зайдля для последовательностей)}%
	\addtocounter{nteor}{5}%
	Пусть $u_n \colon X \to \rr$ --- непрерывны в точке $x_0 \in X$ \uns{($X$ --- метрическое пространство)} и  $\sum\limits_{n = 1}^{\infty}u_n(x)$ сходится равномерно к функции $S(x)$. Тогда $S(x)$ --- непрерывна в точке $x_0$
\end{teor} % Конец теоремы Стокса-Зайдля для рядов

\begin{prf} % Доказательство теоремы Стокса-Зайдля для рядов
	Частичная сумма $S_k(x) = \sum\limits_{n = 1}^{k}u_n(x)$ --- непрерывна в точке $x_0$ и\begin{tikzcd}
		S_k(x) \ar[yshift=2pt]{r}{X}
		\ar[yshift=-2pt]{r}[swap]{k \to \infty} & S(x)
	\end{tikzcd}\! по определению равномерной сходимости ряда (\ref{опр:равн.сх.ряда}). Тогда по теореме Стокса-Зайдля для последовательностей (т. \ref{ст-зд}) $S(x)$ непрерывна в точке $x_0$. 
\end{prf} % Конец доказательства теоремы Стокса-Зайдля для рядов

\begin{teor}[https://www.youtube.com/live/g4Zgeu8xe-Q?si=GEu4QG3yGOvet_UK&t=10772]{об интегрировании функционального ряда}[]*%
	\addtocounter{nteor}{-3}%
	\addcontentsline{toc}{subsection}{\textcolor[rgb]{1, 0.478, 0.612}{\arabic{nteor}}\quad Следствие для рядов (интегрирование функционального ряда)}%
	\addtocounter{nteor}{3}%
	Пусть $u_n \in C[a, b]$ \uns{($u_n \colon [a, b] \to \rr$)}, ряд $\sum\limits_{n = 1}^{\infty}u_n(x)$ сходится равномерно к функции $S(x)$, тогда \[\sum_{n = 1}^{\infty} \int_a^b u_n(x) = \int_a^b S(x)\]
\end{teor} % Конец теоремы об интегрировании функционалного ряда

\begin{prf} % Доказательстсво теоремы об интегрировании функционального ряда
		Частичная сумма $S_k(x) = \sum\limits_{n = 1}^{k}u_n(x) \in C[a, b]$ и\begin{tikzcd}
			S_k(x) \ar[yshift=2pt]{r}{X}
			\ar[yshift=-2pt]{r}[swap]{k \to \infty} & S(x)
		\end{tikzcd}\!
		по определению равномерной сходимости ряда (\ref{опр:равн.сх.ряда}). Тогда по теореме \ref{пер.под инт.} $\int_a^b S_k(x) \xrightarrow[k \to \infty]{} \int_a^b S(x)$. 
		Значит, делая предельный переход при $k \to \infty$ в равенстве $\sum\limits_{n = 1}^{k}\int_a^b u_n(x) = \int_a^b S_k(x)$, получаем доказываемую формулу. \uns{\hypersetup{linkcolor=mygray}
		По теореме \ref{ст-зд} $\int_a^bS(x)$ имеет смысл, т.к. $S_K$ непрерывны и сходятся равномерно к $S(x)$.\linebreak}
\end{prf} % Конец доказательства теоремы об интегрировании функционального ряда

\begin{teor}[https://www.youtube.com/live/g4Zgeu8xe-Q?si=VKzpWKydqsZrl_u4&t=12003]{о дифференцировании ряда}[]*\label{дифф.ряда}%
	\addtocounter{nteor}{-1}%
	\addcontentsline{toc}{subsection}{\textcolor[rgb]{1, 0.478, 0.612}{\arabic{nteor}}\quad Дифференцирование функционального ряда}%
	\addtocounter{nteor}{1}%
	Пусть $u_n\in C^1 \scal{a, b}$, ряд $\sum\limits_{n = 1}^{\infty} u_n(x)$ сходится поточечно к $S(x)$ на \scal{a, b}, и $\sum\limits_{n = 1}^{\infty} u'_n(x)$ равномерно сходится к $\varphi(x)$ на \scal{a, b}. Тогда $S(x) \in C^1\scal{a,b}$ и $S'(x) = \varphi$.
\end{teor} % Конец теоремы  о дифференцировании ряда

\begin{prf} % Доказательство теоремы о дифференцировании ряда
	Частичная сумма $S_k(x) = \sum\limits_{n = 1}^{k}u_n(x) \in C^1\scal{a, b}$ и сходится поточечно к $S(x)$ по определению поточечной сходимости ряда (\ref{опр:равн.сх.ряда}), а последовательность функций \raisebox{0pt}[11pt][14pt]{$\sum\limits_{n = 1}^{k} u'_n(x)$} сходится равномерно к $\varphi(x)$, значит из теоремы \ref{пер.под призв.} получаем $S(x) \in C^1\scal{a, b}$ и $S'(x) = \varphi(x)$. 
\end{prf} % Конец доказательства о дифференцировании ряда