\documentclass[a4paper, 12pt]{article} % Установка размера бумаги, размера шрифта и типа документа

% Пакеты для работы русского языка
\usepackage[english, russian]{babel}
\usepackage[T2A]{fontenc}
\usepackage[utf8]{inputenc}

\usepackage[usenames]{color} % Пакет для цветного текста и фонов
\usepackage{xcolor} % Ещё пакет для цвета
\usepackage{hyperref} % Пакет для ссылок
\usepackage{xfrac}

\usepackage{pifont}

\definecolor{mygray}{rgb}{0.52,0.52,0.52}
\definecolor{linkcolor}{rgb}{0, 0, 0} % цвет ссылок
\definecolor{urlcolor}{rgb}{0, 0.2, 0.49} % цвет гиперссылок
\definecolor{urlcolorin}{rgb}{0, 0, 0} % цвет гиперссылок в цветных прямоугольниках
\hypersetup{pdfstartview=FitH,  linkcolor=linkcolor,urlcolor=urlcolor, colorlinks=true}

\usepackage[babel=true,tracking=true]{microtype}
\SetTracking{encoding = *}{100}

\parindent=0pt % Без абзацного отступа

\usepackage{amsmath, amsfonts, amssymb, amsthm, mathtools} % Пакеты от Американского математического общества (ams) - нужны

\usepackage{geometry} % Пакет для изменения размеров полей
\geometry{top = 15mm}
\geometry{bottom = 15mm}
\geometry{left = 15mm}
\geometry{right = 10mm}

\usepackage{titleps} % Пакет для колонтитулов
\newpagestyle{main}{ % Создаём стиль с названием main
	\setfoot{}{}{\thepage} % Текст снизу (page - счётчик страниц)
}

\pagestyle{main} % Установка этого стиля

\usepackage{enumitem} % Для изменения пометок в списках через необязательные параметры
\usepackage{multicol} % Для набора текста в несколько колонок

\usepackage{tikz} 
% Команда для символов в кружочке
\newcommand*\circled[1]{\tikz[baseline=(char.base)]{
		\node[shape=circle,draw,inner sep=1.5pt] (char) {\footnotesize#1};}}
\newcommand*\circl[1]{\tikz[baseline=(char.base)]{
		\node[shape=circle,draw,inner sep=.5pt] (char) {\tiny#1};}}
	
\usepackage{tikz-cd}
	
\renewcommand{\labelenumii}{\theenumii)} % Переопределение вида нумерации в нумерованном списке на втором уровне вложенности

\makeatletter % Делает @ буквой

% Новые команды
\newcommand*{\rr}{\ensuremath{\mathbb R}}
\newcommand*{\cc}{\ensuremath{\mathbb{C}}}
\newcommand*{\eqdef}{\ensuremath{\stackrel{\mathrm{def}}{=}}}
\newcommand*{\eps}{\ensuremath{\varepsilon}}
\newcommand*{\A}{\ensuremath{\forall \,}}
\newcommand*{\E}{\ensuremath{\hspace{7pt plus 1pt minus 2pt} \exists \,}}
\newcommand*{\eq}{\ensuremath{\Leftrightarrow}}
\newcommand*{\<}{\ensuremath{\leqslant}}
\newcommand*{\biger}{\ensuremath{\geqslant}}
\newcommand*{\scal}[1]{\ensuremath{\left<#1\right>}}
\newcommand*{\bscal}[1]{\ensuremath{\bigl<#1\bigr>}}
\newcommand{\uns}[1]{\textcolor[rgb]{0.52,0.52,0.52}{#1}}
\newcommand*{\rselection}[1]{\textcolor[rgb]{1, 0.318, 0.318}{#1}}
\newcommand*{\rtext}[1]{\textcolor[rgb]{1, 0.463, 0.463}{#1}}

\newcommand{\ybox}[1]{\colorbox[rgb]{0.973, 1, 0.588}{#1}}
\newcommand*{\urlybox}[2]{\hypersetup{urlcolor=urlcolorin}\colorbox[rgb]{0.973, 1, 0.588}{\href{#1}{#2}}\hypersetup{urlcolor=urlcolor}}
\newcommand*{\metr}[2][]{\ensuremath{\rho_{#1}\left(#2\right)}}
\newcommand*{\bmetr}[2][]{\ensuremath{\rho_{#1}\bigl(#2\bigr)}}
\newcommand*{\B}[1]{\ensuremath{\mathrm B(#1)}}
\newcommand*{\BB}[1]{\ensuremath{\mathrm B\bigl(#1\bigr)}}
\newcommand*{\no}{\ensuremath{\varnothing}}
\newcommand*{\rproof}{\fboxsep=2pt\raisebox{1pt}[9pt][1pt]{\fbox{\raisebox{-.5pt}[1ex][0pt]{\ensuremath{\Rightarrow}}}}}
\newcommand*{\lproof}{\fboxsep=2pt\raisebox{1pt}[9pt][1pt]{\fbox{\raisebox{-.5pt}[1ex][0pt]{\ensuremath{\Leftarrow}}}}}
\newcommand*{\rmm}{\ensuremath{\mathbb R^m}}
\newcommand*{\vp}[2][]{\ensuremath{\big|_{#2}^{#1}}}
\newcommand*{\pfrac}[2]{\ensuremath{\frac{\partial #1}{\partial #2}}}
\newcommand*{\Int}{\ensuremath{\mathrm{Int}\,}}
\newcommand*{\unbr}[2]{\uns{\ensuremath{\underbrace{\color[rgb]{0,0,0}{#1}}_{#2}}}}
\newcommand*{\rom}[1]{\expandafter\@slowromancap\romannumeral #1@}
\renewcommand{\thesubsection}{\arabic{subsection}}
\newcommand*{\grad}{\ensuremath{\mathrm{grad\,}}}

\newlength{\shirina} % Параметр \shirina со здачением длины
\newlength{\raznost} % Параметр \raznost со здачением длины
\newcommand{\palka}[1]{% Новая команда
~\\[-5pt plus 2pt minus 1pt]% "Отступ"
\shirina=0.9\textwidth% Присваеваем в \shirina 90% от ширины текста
\settowidth{\raznost}{\vrule\hspace{.5em}}% Присваемваем в \raznost сколько места занимает линейка и отступ
\addtolength{\shirina}{-\raznost}% Вычитаем \raznost из \shirina
\noindent\hbox{%
	\vrule\hspace{.5em}\parbox{\shirina}{#1}}% Сам абзац и линейка
\vspace{5pt plus 2 pt minus 1pt}% Отступ после
}

\renewcommand{\@listI}{% Команда, которая выполняется при входе в списки
	\leftmargin=20pt % Отступ слева 20pt
	\addtolength{\leftmargin}{\hangindent} % Плюс значение, которое установлено для дополнительного отступа у всех строк абзаца
	\rightmargin=0pt % Без отсутпа справа
	\labelsep=5pt % Расстояние между концом заколовка элемента списка и самим элементом
	\labelwidth=20pt % Место по горизонтали, отводимое для заголовка по умолчанию
	\itemindent=0pt % Без дополнительного сдвига заголовка вправо
	\topsep=8pt plus 2pt minus 4pt % Вертикальные отступы до и после списка
	\partopsep=2pt plus 1pt minus 1pt % Дополнительные вертикальные отступы до и после списка, если список начинает абзац
	\parsep=2pt plus 1pt minus 2pt % Вертикальный отступ между абзацами внутри одного элемента списка
	\itemsep=3pt plus 1pt minus 2pt % Расстояние по вертикали, отделяющее элементы списка (дополнительно к \parsep)
}


\newcounter{opr} % Создание нового счётчика opr
\renewcommand{\p@opr}{опр.~} % Изменение ссылочного префикса для счётчика opr
\newenvironment{opr}{% Создание нового окружения opr (для определений)
	\refstepcounter{opr}% Увеличение счётчика opr на 1
	\par % Новый абзац 
	\vspace{7pt plus 2pt minus 2pt}% Отступ до определения
	
	\hangindent=20pt% Дополнительный отступ по горизонтали всех строк абзаца, кроме первой
	\colorbox[rgb]{0.973, 1, 0.588}{\sffamily\itshape Определение \theopr:}% Текст на жёлтом фоне
}
{\par\vspace{5pt plus 2pt minus 3pt}} % Новый абзац и отступ


\newcounter{utv} % Создание нового счётчика utv
\renewcommand{\p@utv}{утв.~} % Изменение ссылочного префикса для счётчика utv
\NewDocumentEnvironment{utv}{o}{% Создание нового окружения utv (для утверждений)
	\refstepcounter{utv}% Увеличение счётчика utv на 1 
	\par % Новый абзац
	\vspace{7pt plus 2pt minus 2pt}% Отступ до утверждения
	\hangindent=20pt % Дополнительный отступ по горизонтали всех строк абзаца, кроме первой
	\IfNoValueTF{#1}%
	{\colorbox[rgb]{0.718,0.984,1}{\sffamily\itshape Утверждение \theutv:}}% Текст на голубом фоне, если нет необязательного аргумента 
	{%
	\hypersetup{urlcolor=urlcolorin}% Ссылка без цвета
	\colorbox[rgb]{0.718,0.984,1}{\sffamily\itshape\href{#1}{Утверждение \theutv:}}% Текст с сылкой на голубом фоне, если есть необязательный аргумент
	\hypersetup{urlcolor=urlcolor}% Возвращение цвета ссылок
	}%
}
{\par\vspace{5pt plus 2pt minus 2pt}} % Новый абзац и отступ


\newcounter{slv} % Создание нового счётчика slv 
\renewcommand{\p@slv}{сл.~}
\NewDocumentEnvironment{slv}{o}{% Создание нового окружения slv (для следствий)
	\refstepcounter{slv} % Увеличение счётчика slv на 1
	\par % Новый абзац 
	\vspace{7pt plus 2pt minus 2pt} % Отступ до следствия
	\hangindent=20pt % Дополнительный отступ по горизонтали всех строк абзаца, кроме первой
	\IfNoValueTF{#1}%
	{\colorbox[rgb]{0.89,0.714,1}{\sffamily\itshape Следствие \theslv:}}% Текст на фиолетовом фоне, если нет необязательного аргумента
	{%
	\hypersetup{urlcolor=urlcolorin}% Ссылка без цвета
	\colorbox[rgb]{0.89,0.714,1}{\sffamily\itshape\href{#1}{Следствие \theslv:}}% Текст с сылкой на голубом фоне, если есть необязательный аргумент
	\hypersetup{urlcolor=urlcolor}% Возвращение цвета ссылок
	}%
}
{\par\vspace{5pt plus 2pt minus 2pt}} % Новый абзац и отступ
 
 
\newenvironment{prf}{% Создание нового окружения prf (для доказательств)
	\par % Новый абзац
	\vspace{7pt plus 2pt minus 2 pt}
	\hangindent=20pt % Дополнительный отступ по горизонтали всех строк абзаца, кроме первой
	\colorbox[rgb]{0.718,1,0.769}{\sffamily\itshape Доказательство:}% Текст на зелёном фоне
}
{\qed\par\vspace{5pt plus 2pt minus 2pt}} % Квадратик, который обозначает конец доказательства, и новый абзац


\newcounter{teor} % Создание нового счётчика teor
\NewDocumentEnvironment{teor}{o m}{% Создание нового окружения teor (для теорем)
	\refstepcounter{teor} % Увеличение счётчика teor на 1
	\par % Новый абзац 
	\vspace{7pt plus 2pt minus 3pt} % Отступ до теоремы
	\hangindent=20pt % Дополнительный отступ по горизонтали всех строк абзаца, кроме первой
	\IfNoValueTF{#1}%
	{\colorbox[rgb]{1, 0.784, 0.376}{\sffamily\itshape Теорема \theteor\normalfont\textit{ (#2)}\sffamily\itshape:}}% Текст на оранжевом фоне, если нет необязательного аргумента
	{%
		\hypersetup{urlcolor=urlcolorin}% Ссылка без цвета
		\colorbox[rgb]{1, 0.784, 0.376}{\sffamily\itshape\href{#1}{{Теорема \theteor \normalfont\textit{ (#2)}\sffamily\itshape:}}}% Текст с сылкой на оранжевом фоне, если есть необязательный аргумент
		\hypersetup{urlcolor=urlcolor}% Возвращение цвета ссылок
	}%
	\\[4pt plus 1pt minus 2pt]% Новая строка
}
{\par\vspace{5pt plus 2pt minus 2pt}} % Новый абзац и отступ


\newcounter{zam} % Создание нового счётчика zam
\renewcommand{\p@zam}{зам.~} % Изменение ссылочного префикса для счётчика zam
\NewDocumentEnvironment{zam}{o}{% Создание нового окружения zam (для замечаний)
	\refstepcounter{zam} % Увеличение счётчика zam на 1
	\vspace{7pt plus 2pt minus 3 pt}% Отступ до замечания
	\par % Новый абзац 
	\hangindent=20pt % Дополнительный отступ по горизонтали всех строк абзаца, кроме первой
	\IfNoValueTF{#1}
	{\colorbox[rgb]{1,0.745,0.953}{\sffamily\itshape Замечание \thezam:}}% Текст на розовом фоне, если нет необязательного аргумента
	{%
	\hypersetup{urlcolor=urlcolorin}% Ссылка без цвета
	\colorbox[rgb]{1,0.745,0.953}{\sffamily\itshape\href{#1}{Замечание \thezam:}}% Текст c ссылкой на розовом фоне, если есть необязательный аргумент
	\hypersetup{urlcolor=urlcolor}% Возвращение цвета ссылок
	}%
}
{\par\vspace{5pt plus 2pt minus 2 pt}} % Новый абзац и отступ
 

\newcounter{lem} % Создание нового счётчика teor
\NewDocumentEnvironment{lem}{o m}{% Создание нового окружения teor (для теорем)
	\refstepcounter{lem} % Увеличение счётчика teor на 1
	\par % Новый абзац 
	\vspace{7pt plus 2pt minus 3pt} % Отступ до теоремы
	\hangindent=20pt % Дополнительный отступ по горизонтали всех строк абзаца, кроме первой
	\IfNoValueTF{#1}%
	{\colorbox[rgb]{0.82,1,0.58}{\sffamily\itshape Лемма \thelem\normalfont\textit{ (#2)}\sffamily\itshape:}}% Текст на оранжевом фоне, если нет необязательного аргумента
	{%
		\hypersetup{urlcolor=urlcolorin}% Ссылка без цвета
		\colorbox[rgb]{0.82,1,0.58}{\sffamily\itshape\href{#1}{{Лемма \thelem \normalfont\textit{ (#2)}\sffamily\itshape:}}}% Текст с сылкой на оранжевом фоне, если есть необязательный аргумент
		\hypersetup{urlcolor=urlcolor}% Возвращение цвета ссылок
	}%
	\\[4pt plus 1pt minus 2pt]% Новая строка
}
{\par\vspace{5pt plus 2pt minus 2pt}} % Новый абзац и отступ

\makeatother % Перестаём считать @ буквой

\hfuzz=2pt
