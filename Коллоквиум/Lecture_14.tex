\subsection*{\S\ Степенные ряды}

\begin{opr}
	Пусть $z_0 \in \cc,\ r \in (0, +\infty)$, тогда $\B{z, r} = \{\,z \in \cc \mid |z - z_0| < r \,\} \uns{{}\subset \cc}$ называется \urlybox{https://youtu.be/zgKkH3Nr6-4?si=xUG7X81RzS13Xsfz&t=1919}{кругом} c центром в точке $z_0$ и радиуса $r$.
\end{opr}

\begin{opr}\add*{Степенной ряд, радиус сходимости степенного ряда, формула Адамара}
	Пусть $a_n$ --- комплексная последовательность, $z_0 \in \cc$, тогда $A(z) = \sum\limits_{n = 0}^{\infty} a_n(z - z_0)^n$ называется \urlybox{https://youtu.be/zgKkH3Nr6-4?si=ldfvxbmy3ue68QgP&t=1971}{степенным рядом} \uns{\small($A \colon E \subset \cc \to \cc$)}. Ряд \urlybox{https://youtu.be/zgKkH3Nr6-4?si=cGk4nMm0Rx1cQrFJ&t=2001}{сходится}, если сходится последовательность частичных сумм. \urlybox{https://youtu.be/zgKkH3Nr6-4?si=aSiEvDMGSJV_7FoE&t=2453}{Сходится абсолютно}, если сходится вещественный ряд $\sum\limits_{n = 0}^{\infty} |a_n(z - z_0)^n|$
\end{opr}

\begin{zam}
	Комплексный ряд \raisebox{0pt}[0pt][5pt]{$\sum\limits_{n = 0}^{\infty} a_n$} сходится \eq\ сходятся вещественные ряды \raisebox{0pt}[0pt][0pt]{$\sum\limits_{n = 0}^{\infty} \mathrm{Re}\, a_n$} и $\sum\limits_{n = 0}^{\infty} \mathrm{Im}\, a_n$ \uns{\hypersetup{linkcolor=mygray}(это покоординатная сходимость, \ref{покоорд.сход.})}. 
	Если ряд $\sum\limits_{n = 0}^{\infty} |a_n|$ сходится, то и $\sum\limits_{n = 0}^{\infty} a_n$ сходится, так как |$\mathrm{Re}\, a_n| \< |a_n|$ и $|\mathrm{Im}\, a_n| \< |a_n|$, значит ряды из вещественных и мнимых \medskip частей сходятся абсолютно $\Rightarrow$ сходятся. 
\end{zam}

\begin{teor}[https://youtu.be/zgKkH3Nr6-4?si=MvlhhJxcgF8KG0Mf&t=2204]{о круге сходимости степенного ряда}\label{круг сходимости}
	Пусть $A(z) = \sum\limits_{n = 0}^{\infty} a_n(z - z_0)^n$ --- степенной ряд, тогда возможно
	\begin{enumerate}
		\item Ряд сходится абсолютно при любом $z \uns{{} \in \cc}$
		
		\item Ряд сходится абсолютно только при $z = z_0$, иначе расходится
		
		\item \hspace{-5pt}$\E R \in (0, +\infty) : {}$ ряд сходится абсолютно при $|z - z_0| < R$ и расходится при $|z - z_0| > R$ 
	\end{enumerate}
\end{teor}

\begin{prf}\\
	\palka{\small\begin{enumerate}[leftmargin=15pt]
		\item Верхний предел вещественной последовательности $x_n$ это \[\varlimsup\limits_{n \to \infty} x_n = \lim\limits_{n \to \infty} y_n \text{,\qquad где } y_n = \sup\{\, x_n, x_{n+1}, x_{n+ 2}, \dots \,\}\]
		
		\item \textit{Признак Коши сходимости вещественного ряда.} Пусть $\sum\limits_{n = 1}^{\infty} x_n$ --- неотрицательный вещественный ряд, тогда он сходится, если $\varlimsup\limits_{n \to \infty} \sqrt[n]{x_n} < 1$, и расходится, если $\varlimsup\limits_{n \to \infty} \sqrt[n]{x_n} > 1$, причём $x_n \xrightarrow[n\to \infty]{}0$ \raisebox{-8pt}[0pt][0pt]{\makebox[0pt]{\hspace{-55pt} \begin{picture}(10,10)\thicklines\put(0,0){\line(1,2){9}}\end{picture}}}
		\end{enumerate}}\\
	По признаку Коши ряд $\sum\limits_{n = 0}^{\infty} |a_n(z - z_0)^n|$ сходится, если 
	\[\varlimsup\limits_{n \to \infty} \sqrt[n]{|a_n(z - z_0)^n|} \uns{{} = \varlimsup\limits_{n \to \infty} |z - z_0| \cdot \sqrt[n]{|a_n|}} = |z - z_0| \cdot \varlimsup\limits_{n \to \infty} \sqrt[n]{|a_n|} < 1\] То есть может быть 3 случая:
	\begin{enumerate}
		\item Если $\varlimsup\limits_{n \to \infty} \sqrt[n]{|a_n|} = 0$, то ряд $A$ сходится при любом $z \uns{{} \in \cc}$ 
		
		\item Если $\varlimsup\limits_{n \to \infty} \sqrt[n]{|a_n|} = +\infty$, то ряд $A$ сходится только при $z = z_0$, и расходится в остальных случаях, т.к. слагаемые не стремятся к 0
		
		\item Если $|z - z_0| <{}$ {\small$\cfrac{1}{\varlimsup\limits_{n \to \infty} \sqrt[n]{|a_n|}}$}, то ряд $A$ сходится абсолютно
	\end{enumerate}
	\hspace{20pt}Таким образом  в последнем пункте $R ={}$ {\small$\cfrac{1}{\varlimsup\limits_{n \to \infty} \sqrt[n]{|a_n|}}$} --- это формула Коши-Адамара.
\end{prf}

\begin{zam}[https://youtu.be/zgKkH3Nr6-4?si=DDcTOSeonrxbvb41&t=3079]
	Был ещё \textit{признак Даламбера: Пусть $\sum\limits_{n = 1}^{\infty} x_n$ --- положительный вещественный ряд и$\E \lim\limits_{n \to \infty}\frac{x_{n + 1}}{x_n} = D$, тогда ряд сходится, если $D < 1$, и расходится, если $D > 1$.} Поэтому в теореме \ref{круг сходимости} радиус круга сходимости можно считать по формуле $R = \lim\limits_{n \to \infty}\frac{|a_n|}{|a_{n + 1}|}$ (если этот предел существует).
\end{zam}

\begin{teor}[https://youtu.be/zgKkH3Nr6-4?si=Reb0bWLxxqPReBy9&t=3976]{о равномерной сходимости и непрерывности степенного ряда}
		Пусть $f(z) = \sum\limits_{n = 0}^{\infty} a_n(z - z_0)^n$ --- степенной ряд и $R$ --- радиус его круга сходимости. Тогда:
		\begin{enumerate}
			\item $\A r \in (0, R)$\quad $\sum\limits_{n = 0}^{\infty} a_n(z - z_0)^n$ равномерно сходится на $\overline{\B{z_0, r}}$
			
			\item $f(z)$ непрерывна на $\B{z_0, R}$
		\end{enumerate}
\end{teor}

\begin{prf}\begin{enumerate}
	\item Так как на $\overline{\B{z_0, r}}$ $ a_n(z - z_0)^n \< a_n r^n$ и ряд \raisebox{0pt}[0pt][5pt]{$\sum\limits_{n = 0}^{\infty} a_n r^n$} сходится на $\overline{\B{z_0, r}}$\uns{, потому что $\sum\limits_{n = 0}^{\infty} a_n r^n = \sum\limits_{n = 0}^{\infty}a_n\bigl((z_0 + r) - z_0\bigr)$ и $(z_0 + r) \in \B{z_0, R}$}, то по признаку Вейерштрасса (теорема~\ref{пр.вейер.}) на $\overline{\B{z_0, r}}$ равномерно сходится ряд \raisebox{0pt}[10pt][10pt]{$\sum\limits_{n = 0}^{\infty} a_n(z - z_0)^n$}.
	
	\item $\A z \in \B{z_0, R}$ возьмём $r \in \bigl(|z - z_0|, R\bigr)$, тогда в шаре $\overline{\B{z_0, r}}$ есть равномерная сходимость ряда $\sum\limits_{n = 0}^{\infty} a_n(z - z_0)^n$ \uns{(из пункта 1)}, и $a_n(z - z_0)^n$ непрерывна, значит по теореме Стокса-Зайдля (теорема \ref{ст-зд ряды}) $f$ непрерывна в точке $z$.
\end{enumerate}\end{prf}

\begin{lem}[https://www.youtube.com/live/AWZbCBfOlt4?si=T48B5aabxehlgrPt&t=11387]{просто}*\label{просто}
	Если $w, w_0 \in \cc$,$\uns{\E r \in \rr :{}} |w|, |w_0| \< r$, тогда \uns{$\A n \in \mathbb N$} $|w^n - w_0^n| \< n \cdot r^{n - 1} \cdot |w - w_0|$
\end{lem}

\begin{prf}
	Раскладываем на множители \uns{(и пользуемся неравенством треугольника)}:\\
	$|w^n - w_0^n| = \bigl|(w - w_0) \cdot (w^{n - 1} + w^{n - 2} \cdot w_0 + w^{n - 3} \cdot w_0^2 + \ldots + w \cdot w_0^{n - 2} + w_0^{n - 1})\bigr| \<{}$ \\
	${}\<|w - w_0| \cdot \bigl(|w^{n - 1}| + |w^{n - 2} \cdot w_0| + |w^{n - 3} \cdot w_0^2| + \ldots + |w \cdot w_0^{n - 2}| +|w_0^{n - 1}|\bigr) \< |w - w_0| \cdot n \cdot r^{n - 1}$
\end{prf}

\begin{teor}[https://www.youtube.com/live/F2g5eXOh4dk?si=duGVAZL42TPq1fYe&t=502]{о дифференцировании степенного ряда}
	Пусть радиус круга сходимости степенного ряда $f(z) = \sum\limits_{n = 0}^{\infty} a_n (z-z_0)^n$ равен $R$, тогда радиус круга сходимости ряда \raisebox{0pt}[0pt][0pt]{$\sum\limits_{n = 1}^{\infty} n \cdot a_n (z-z_0)^{n - 1}$} тоже равен $R$ и \raisebox{0pt}[12pt][0pt]{$f'(z) = \sum\limits_{n = 1}^{\infty} n \cdot a_n (z-z_0)^{n - 1}$}
\end{teor}

\begin{prf}\begin{enumerate}
	\item Радиусы кругов сходимости рядов $\sum\limits_{n = 1}^{\infty} n \cdot a_n (z-z_0)^{n - 1}$ и $\sum\limits_{n = 1}^{\infty} n \cdot a_n (z-z_0)^{n}$ одинаковые 
	\uns{(при $z = t$ частичные суммы первого ряда сходятся к $g$ \eq\ частичные суммы второго ряда сходятся к $(t - z_0)g$)}, тогда по формуле Коши-Адамара (т. \ref{круг сходимости}) радиус круга сходимости этого ряда равен $(\varlimsup\limits_{n \to \infty} \sqrt[n]{|a_n| \cdot n})^{-1} = (\varlimsup\limits_{n \to \infty} \sqrt[n]{|a_n|})^{-1} = R$, \uns{т.к. $\varlimsup\limits_{n \to \infty} \sqrt[n] n = 1$} 
	
	\item Будем искать производную \uns{(комплексную)} $f$ в точке $a \in \B{z_0, R}$ по определению:
	\[\frac{f(z) - f(a)}{z - a} = \sum_{n = 0}^{\infty} a_n \frac{(z - z_0)^n - (a - z_0)^n}{z - a} \quad\text{\uns{--- подставили $f$}}\]
	Пусть $w = z - z_0$, $w_0 = a - z_0$, тогда $w - w_0 = z - a$, значит в шаре $\B{z_0, r}$ ($r < R$ такое, что $|w|, |w_0| \< r$) по лемме \ref{просто} 
	\[\left| a_n \frac{w^n - w_0^n}{w - w_0}\right| \< a_n \cdot n \cdot r^{n - 1} \] 
	Тогда по признаку Вейерштрасса (т. \ref{пр.вейер.}) ряд $\sum\limits_{n = 1}^{\infty} a_n \frac{w^n - w_0^n}{w - w_0}$ сходится, т.к. ряд $\sum\limits_{n = 1}^{\infty} a_n \cdot n \cdot r^{n - 1}$\linebreak сходится, потому что это ряд \raisebox{0pt}[10pt]{$\sum\limits_{n = 1}^{\infty} n \cdot a_n (z-z_0)^{n - 1}$}, в который подставлено $z = z_0 + r \uns{{}\in \B{z_0, R}}$, а он сходится по пункту 1 (и сходится абсолютно по т. \ref{круг сходимости}). Значит по теореме \ref{пред.пер.в сум.}
	\[\lim_{w \to w_0}\sum_{n = 1}^{\infty} a_n \frac{w^n - w_0^n}{w - w_0} = \sum_{n = 1}^{\infty} \lim_{w \to w_0} a_n \frac{w^n - w_0^n}{w - w_0} = \sum_{n = 1}^{\infty}  a_n \cdot n \cdot w_0^{n - 1}\]
	Последнее равенство получается, раскладывая $w^n - w_0^n$ на множители как в лемме \ref{просто} и сокращая $w - w_0$. B подставляя то, чему равно $w_0$, получаем, что
	\[f'(a) = \sum_{n = 1}^{\infty} n \cdot a_n (a - z_0)^{n - 1}\]
\end{enumerate}\end{prf}

\begin{slv}[https://www.youtube.com/live/F2g5eXOh4dk?si=d9I5R8WwiUjJdq_V&t=1918]
	\begin{enumerate}
		\item Если радиус круга сходимости степенного ряда $f(z) = \sum\limits_{n = 0}^{\infty} a_n (z-z_0)^n$ равен $R$, тогда $f \in C^{\infty}\bigl(\B{z_0, R}\bigr)$ \uns{(после дифференцирования степенного ряда получается степенной ряд, который можно опять дифференцировать)}\pagebreak
		
		\item Для вещественных степенных рядов $f(x) = \sum\limits_{n = 0}^{\infty} a_n (x - x_0)^n$ (т.е. $a_n, z, z_0 \in \rr$) выполнено:
		\[\int_{x_0}^x f(t) \,dt = \sum_{n = 0}^{\infty} \frac{a_n}{n + 1}(x - x_0)^{n + 1}\]
		Потому что беря производную от этого ряда, получаем $f(x)$. Слева и справа написаны первообразные функции $f$, они могут отличаться на константу, но в точке $x_0$ они обе равны нулю, значит они совпадают.\\[5pt]
		\textit{Пример:} $-\frac 1 {1 + x^2} = -1 + x^2 - x^4 + x^6 - \ldots$\quad \uns{(сумма геометрической прогрессии)}, тогда
		\[\int_{0}^x -\frac 1 {1 + t^2} \,dt = - x + \frac{x^3}{3} - \frac{x^5}{5} + \ldots \quad \text{\small(под интегралом стоит производная $\arcctg t$), значит}\] 
		\[\arcctg t \bigg|^x_0 = \arcctg x - \frac{\pi}{2} \quad \Rightarrow \quad \arcctg x = \frac{\pi}{2} - x + \frac{x^3}{3} - \frac{x^5}{5} + \ldots\] 
	\end{enumerate}
\end{slv}
