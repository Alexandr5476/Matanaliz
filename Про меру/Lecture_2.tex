\subsection*{\S\ Мера}

\begin{opr}\label{опр.меры}
	Функция $\mu \colon \p \to \overline\rr$ называется \urlybox{https://www.youtube.com/live/OdDauqCjZt0?si=_fjwqMmuHsaBpBbR&t=11075}{мерой} (\p --- полукольцо), если она является объёмом, и если она\!\! \urlybox{https://www.youtube.com/live/OdDauqCjZt0?si=_fjwqMmuHsaBpBbR&t=11075}{счётно-аддитивна}\!, то есть $\A A_1, A_2, A_3 \ldots \in \p$ (дизъюнктные), если $\bigsqcup\limits_{i = 1}^\infty A_i \in \p$, то\linebreak \[\mu\left(\bigsqcup\limits_{i = 1}^\infty A_i\right) = \sum\limits_{i = 1}^\infty \mu(A_i)\]
\end{opr}

\begin{teor}[https://www.youtube.com/live/OdDauqCjZt0?si=Fmsn6BXwN-FG8ONz&t=11826]{об эквивалентности счетной аддитивности и счетной полуаддитивности}\label{полуадд.меры}
	Пусть  $\mu \colon \p \to \overline\rr$ --- объём (\p --- полукольцо), тогда эквивалентно:
	\begin{enumerate}
		\item $\mu$ --- мера (т.е. $\mu$ --- счётно-аддитивна)
		
		\item\label{сч-полуадд.} $\mu$ --- счётно-полуаддитивна, т.е. $\A A, A_1, A_2 \ldots \in \p$, если $A \subset \bigcup\limits_{i = 1}^\infty A_i$, то $\mu(A) \<  \sum\limits_{i = 1}^\infty \mu(A_i)$
	\end{enumerate}
\end{teor}

\begin{prf}
	\rpr{1}{2} аналогично доказательству пункта \ref{св-во2объёма} теоремы \ref{св-ва объёма} (заменить конечные суммы и конечные объединения на бесконечные)\\[5pt]
	\lpr{1}{2} Возьмём  $A,{}$\raisebox{0pt}[0pt][0pt]{$ \uns{\underbrace{\textcolor[rgb]{0, 0, 0}{A_1, A_2, \ldots}}_{\text{дизъюнктные}}} \in \p$}${}: A = \bigsqcup\limits_{i = 1}^\infty A_i$, тогда $\A N$ $\bigsqcup\limits_{i = 1}^N A_i \subset A$, и по усиленной монотонности $\sum\limits_{i = 1}^N \mu(A_i) \< \mu(A)$ (свойство \ref{св-во1объёма} объёма), но по условию $\mu(A) \<  \sum\limits_{i = 1}^\infty A_i$, значит $\mu(A) = \sum\limits_{i = 1}^\infty A_i$
\end{prf}

\begin{slv}[https://www.youtube.com/live/OdDauqCjZt0?si=WgYtfSwoHQg006tq&t=12345]\label{объ.мн.меры0}
	Если $A, A_1, A_2, \ldots \in \p$, $A \subset \bigcup\limits_{i = 1}^\infty A_i$, $\A i \uns{{}\in \{\,1, 2, \ldots\,\}}\ \mu(A_i) = 0$, тогда $\mu(A) = 0$ \uns{(\p --- полукольцо, $\mu$ --- мера)}. Это пункт \ref{сч-полуадд.} при $\mu(A_i) = 0$
\end{slv}

\begin{teor}[https://www.youtube.com/live/Ptfcl76lZBs?si=aaO9813UQcucBfHB&t=8902]{о непрерывности меры снизу}\label{непр.меры снизу}
	Пусть $\mathcal A$ --- алгебра, $\mu \colon \mathcal A \to \rr$ --- объём (конечный), тогда эквивалентно:
	\begin{enumerate}
		\item $\mu$ --- мера (т.е. $\mu$ --- счётно-аддитивна)
		
		\item $\A A, A_1, A_2, \ldots \in \mathcal A : A_1 \subset A_2 \subset A_3 \subset \ldots$, $A = \bigcup\limits_{i = 1}^\infty A_i$ $\lim\limits_{n \to \infty} \mu(A_n) = \mu(A)$ 
	\end{enumerate}
\end{teor}

\begin{prf}
	Нужна только формулировка. Доказывается следующая теорема --- о непрерывности меры сверху.
\end{prf}

\begin{teor}[https://www.youtube.com/live/Ptfcl76lZBs?si=aaO9813UQcucBfHB&t=8902]{о непрерывности меры сверху}\label{непр.меры сверху}
	Пусть $\mathcal A$ --- алгебра, $\mu \colon \mathcal A \to \rr$ --- объём (конечный), тогда эквивалентно:
	\begin{enumerate}
		\item $\mu$ --- мера (т.е. $\mu$ --- счётно-аддитивна)
	
		\item $\A A, A_1, A_2, \ldots \in \mathcal A : A_1 \supset A_2 \supset A_3 \supset \ldots$, $A = \bigcap\limits_{i = 1}^\infty A_i$ $\lim\limits_{n \to \infty} \mu(A_n) = \mu(A)$ 
	\end{enumerate}
\end{teor}

\begin{prf}
	\rpr{1}{2} Пусть $\A i  \uns{{}\in \{\,1, 2, \ldots\,\}}$ $B_i = A_i \setminus A_{i+1}$, тогда $\A k\ A_k = A \sqcup \bigsqcup\limits_{i = k}^\infty B_i$ и из счётной аддитивности $\mu(A_k) = \mu(A) + \sum\limits_{i = k}^\infty \mu(B_i)$. Делая предельный переход при $k \to \infty$, получаем, что $\lim\limits_{k \to \infty} \mu(A_k) = \mu(A)$, так как остаток сходящегося ряда стремится к 0 (ряд $\sum\limits_{i = 1}^\infty \mu(B_i)$ сходится, т.к. $\mu(A_1) = \mu(A) + \sum\limits_{i = 1}^\infty \mu(B_i)$ и $\mu(A_1)$ конечен).\\[5pt]
	\lpr{1}{2} Возьмём $C = \bigsqcup\limits_{i = 1}^\infty C_i$ ($C_i \in \mathcal{A}$), тогда для $A_k = \bigsqcup\limits_{i = k + 1}^\infty C_i = C \setminus \bigsqcup\limits_{i = 1}^k C_i \in \mathcal A$ выполнено $A_1 \supset A_2 \supset A_3 \supset \ldots$ и $A = \bigcap\limits_{i = 1}^\infty A_i = \no$ (т.к. если $x \in A$, то $x$ должен принадлежать всем $A_k$ и некоторому $C_N$, но $A_k = C \setminus \bigsqcup\limits_{i = 1}^k C_i$), значит $\mu(C) = \sum\limits_{i = 1}^k \mu(C_i) + \mu(A_k)$. Делая предельный переход при $k \to \infty$ получаем, что $\mu(C) = \sum\limits_{i = 1}^\infty C_i$ (т.к. $\lim\limits_{n \to \infty} \mu(A_n) = \mu(\no) = 0$), то есть $\mu$ счётно-аддитивна.
\end{prf}

\begin{center}
	\textbf{Теорема о продолжении меры}
\end{center}

\begin{opr}
	Мера $\mu \colon \p \to \rr$ (где \p --- полукольцо на множестве $X$), называется \urlybox{https://www.youtube.com/live/Ptfcl76lZBs?si=ASd7DLP6-R3Pcdbm&t=10028}{$\sigma$-конечной}, если$\E P_1, P_2, \ldots \in \p$ такие, что \uns{$\A i \uns{{}\in \{\,1, 2, \ldots\,\}}$} $\mu(P_i)$ --- конечный, и $X = \bigcup\limits_{i = 1}^\infty P_i$
\end{opr}

\begin{opr}\label{полная мера}
	Мера $\mu \colon \mathcal{A} \to \rr$ (где $\mathcal{A}$ --- $\sigma$-алгебра) называется \urlybox{https://www.youtube.com/live/Ptfcl76lZBs?si=dhhkiKFG0wAUQKN4&t=10342}{полной}, если $\A A \in \mathcal{A} : \mu(A) = 0$ выполнено, что $\A B \subset A$ \ $B \in \mathcal{A}$ \uns{(и тогда $\mu(B) = 0$ из-за монотонности объёма)}
\end{opr}
\pagebreak
\begin{teor}[https://www.youtube.com/live/Ptfcl76lZBs?si=5tjUKh1UlMak8EGx&t=10579]{о стандартном (Лебеговоском) продолжении меры}\label{прод.меры}%
	Пусть $\mu_0 \colon \p_0 \to \overline \rr$ --- $\sigma$-конечная мера ($\p_0$ --- полукольцо на множестве $X$), тогда существуют $\sigma$-алгебра $\mathcal A : \p_0 \subset \mathcal A$, и мера $\mu \colon \mathcal A \to \overline \rr$ такая, что $\mu \raisebox{0pt}[0pt][0pt]{\su{\p_0}} = \mu_0$ (т.е. $\mu$ является продолжением $\mu_0$) и у них есть свойства:
	\begin{enumerate}
		\item $\mu$ --- полная
		
		\item $\A \p_1 \subset A$ --- полукольцо${}: \p_0 \subset \p_1$, $\A \mu_1 \colon \p_1 \to \overline{\rr}$ --- мера такая, что $\mu_1 \su{\p_0} = \mu_0$ выполнено, что $\mu\su{\p_1} = \mu_1$ 
		
		\item Если$\E \mathcal A_1$ --- $\sigma$-алгебра, содержащая $A\uns{{}\supset \p_0}$, и если$\E \mu_1 \colon A_1 \to \overline \rr$ --- полная мера такая, что $\mu_1\su{\p_0} = \mu_0$ тогда $\mu_1\su{\mathcal A} = \mu$
		
		\item\label{форм.меры} Для $A \in \mathcal{A}$ \ $\mu(A) = \inf\left\{\sum\limits_{i = 1}^\infty \mu_0 (P_i) \mid P_i \in \p_0 : A \subset \bigcup\limits_{i = 1}^\infty P_i \right\} $
	\end{enumerate}
\end{teor}

\begin{zam}[https://www.youtube.com/live/Ptfcl76lZBs?si=0IEfoZ4GCXaVGxNG&t=11380]
	$\A A \in \mathcal A \E B \in \mathcal{A}$ такое, что $A \subset B$, $\mu(A) = \mu(B)$, $\mu(B \setminus A) = 0$ и $B$ имеет вид $B = \bigcap\limits_{j = 1}^\infty \left( \bigcup\limits_{i = 1}^\infty P_{ij}\right)$, где $P_{ij} \in \p$
\end{zam}

\subsection*{\S\ Мера Лебега}

\begin{lem}[https://www.youtube.com/live/Ptfcl76lZBs?si=oHYtEkpssSIjI8CP&t=11668]{счётная аддитивность классического объема}\label{класс.мера}%
	Стандартный объём $\mu$ на полукольце ячеек $\p^m$ \uns{$\bigl(\A [a,b) \in \p^m\ \mu[a,b) = \prod\limits_{i = 1}^m (b_i - a_i)\bigr)$} является $\sigma$-конечной мерой. 
\end{lem}

\begin{prf}
	Проверим счётную полуаддитивность. Пусть $P = [a, b), P_k = [a_k, b_k)$ --- ячейки такие, что $P \subset \bigcup\limits_{k = 1}^\infty P_k$. Возьмём $\eps, b' : [a, b'] \subset [a, b)$ ; $\mu(P) - \mu[a, b') < \eps$ \uns{(чуть <<уменьшили>> $b$)}, и возьмём $a'_k : [a_k, b_k) \subset (a_k', b_k)$ ; $\mu[a'_k, b_k) - \mu(P_k) < \sfrac{\eps}{2^k}$ \uns{(чуть <<увеличили>> $a$)}, тогда \[[a, b'] \subset \bigcup_{k = 1}^\infty (a_k', b_k)\]
	Так как $[a, b']$ --- компакт, то из его покрытия \raisebox{0pt}[0pt][0pt]{$\bigcup\limits_{k = 1}^\infty (a_k', b_k)$} открытыми множествами, можно выбрать конечное подпокрытие: 
	\[\E N : \uns{[a, b') \subset{}} [a, b'] \subset \bigcup_{k = 1}^N (a_k', b_k) \uns{{}\subset \bigcup_{k = 1}^N [a_k', b_k)}\]
	Объём конечно-полуаддитивен (свойство \ref{св-во2объёма}), значит $\mu[a, b') \< \sum\limits_{k = 1}^N \mu[a_k', b_k)$, т.е. $\mu(P) - \eps \< \eps + \raisebox{0pt}[0pt]{$\sum\limits_{k = 1}^\infty\mu(P_k)$}$, тогда и $\mu(P) \< \raisebox{0pt}[0pt]{$\sum\limits_{k = 1}^\infty\mu(P_k)$}$ \uns{(можно взять $\eps = \sfrac 1n$ и сделать предельный переход при $n \to \infty)$}. Получили, что $\mu$ --- счётно полуаддитивен, значит (по теореме \ref{полуадд.меры}) $\mu$ является мерой. Эта мера $\sigma$-конечна, т.к. счётное объединение, например, всех ячеек со стороной 1 равно \rmm  
\end{prf}

\begin{opr}\label{мера лебега}
	Стандартное продолжение меры (теорема \ref{прод.меры}) на полукольце ячеек (т.е. продолжение классического объёма в \rmm --- замечание \ref{класс.объем} и лемма \ref{класс.мера}) называется \urlybox{https://www.youtube.com/live/FFhHi8qwuDM?si=AP7JyTlDgmEEaRgO&t=6718}{мерой Лебега}. Соответствующая $\sigma$-алгебра обозначается \m, мера Лебега --- $\lambda$. Множества, принадлежащие \m, называются \urlybox{https://www.youtube.com/live/FFhHi8qwuDM?si=AP7JyTlDgmEEaRgO&t=6718}{измеримыми}.
\end{opr}

\begin{zam}[https://www.youtube.com/live/FFhHi8qwuDM?si=E6XVbmkMiVfJAmNk&t=6897]
	Так как $\A a \in \rmm$ $a = \bigcap\limits_{n = 1}^\infty Q(a, \frac 1n)$, где $Q(a, r) = [a_1 - r, a_1 + r) \times \ldots \times [a_m - r, a_m + r)$ и алгебра замкнута относительно пересечений, то $a \in \m$ и $\lambda(a) = 0$, потому что $\lambda\bigl(Q(a, \frac 1n)\bigr) = \left(\frac 2n\right)^m \xrightarrow[n \to \infty]{} 0$, то есть $\inf\left\{\lambda\bigl(Q(a, \frac 1n)\bigr) \mid n \in \mathbb{N}\right\} = 0$ (воспользовались формулой \ref{форм.меры} из теоремы \ref{прод.меры}, рассматривая покрытие одноточечного множества одной ячейкой; получили, что inf по таким покрытиям равен 0, значит inf и по всевозможным покрытиям будет 0). Тогда любое счётное подмножество \rmm\ измеримо, и имеет меру 0, так как если $A_n \in \m, \lambda(A_n) = 0$, то \raisebox{0pt}[13pt][0pt]{$\bigcup\limits_{n = 1}^\infty A_n \in \m$} (замкнутость $\sigma$-алгебры относительно счётного объединения) и $\lambda\left(\bigcup\limits_{n = 1}^\infty A_n\right) = 0$ (следствие \ref{объ.мн.меры0}).
\end{zam}

\begin{lem}[https://www.youtube.com/live/FFhHi8qwuDM?si=01lanTvQNSo3S87X&t=7279]{о структуре открытых множеств и множеств меры 0}
	\vspace{-20pt}
	\begin{enumerate}\makeatletter\renewcommand{\p@enumi}{\thelem.}\makeatother
		\item\label{изм.откр.мн.} Пусть $O \in \rmm$ --- открытое, тогда $O = \bigsqcup\limits_{i = 1}^\infty Q_i$, где $Q_i$ --- рациональная кубическая ячейка (и можно считать, что её координаты --- двоично-рациональные числа, т.е. вида $\frac{n}{2^k}, n \in \mathbb{Z}$, и что $\overline{Q_i} \subset O$ ) 
		
		\item \label{стр.мн.мер.0} Пусть $E \in \m : \lambda(E) = 0$, тогда $\A \eps > 0 \E Q_i$ --- кубические ячейки такие, что \raisebox{0pt}[13pt][0pt]{$E \subset \bigcup\limits_{i = 1}^\infty Q_i$} и $\sum\limits_{i = 1}^\infty\lambda(Q_i) < \eps$ (или$\E B_i$ --- открытые шары $E \subset \bigcup\limits_{i = 1}^\infty B_i$ и $\sum\limits_{i = 1}^\infty\lambda(B_i) < \eps$)
	\end{enumerate}
\end{lem}

\begin{prf}\begin{enumerate}
	\item Для каждого $x \in O$ фиксируем двоично-рациональную ячейку $Q(x) : \overline{Q(x)} \subset O, x \in Q(x)$, тогда $O = \bigcup\limits_{x \in O}Q(x)$, но ячейки рациональные, значит различных ячеек в этом объединении счётное число, поэтому \raisebox{0pt}[0pt][0pt]{$O = \bigcup\limits_{i = 1}^{\infty}Q_i(x)$}. Но эти ячейки пересекаются, сделаем их дизъюнктными: пусть $Q_1 = Q_1(x), Q_2 = Q_2(x) \setminus Q_1 = \bigsqcup\limits_{i = 1}^{l_2} D_i, Q_3 = Q_3(x) \setminus (Q_2 \cup Q_1) = \bigsqcup\limits_{i = 1}^{l_3} E_i$, \dots\linebreak Тогда \raisebox{0pt}[0pt]{$O = Q_1 \sqcup \bigsqcup\limits_{i = 1}^{l_2} D_i \sqcup \bigsqcup\limits_{i = 1}^{l_3} E_i \sqcup \ldots$} (и ячейки $D_i, E_i, \ldots$ являются двоично-рациональными, так как они являются разностью двоично-рациональной ячейки и объединения двоично-рациональных ячеек; такую разность можно представить в виде дизъюнктного объединения двоично-рациональных ячеек, которые тут обозначены $D_i, E_i, \ldots$)
	
	\item По теореме \ref{прод.меры} $\lambda(E) = \inf\left\{\sum\limits_{i = 1}^\infty \lambda(P_i) \mid E \subset \bigcup\limits_{i = 1}^\infty P_i \uns{, P_i \text{ --- ячейки}}\right\}$. Если $\lambda(E) = 0$, то \[\A \eps > 0 \E P_k \text{ --- ячейки} : E \subset \bigcup\limits_{k = 1}^\infty P_k \text{ и } \sum\limits_{k = 1}^\infty \lambda(P_k) < \eps\]
	Ячейки $P_k$ можно покрыть кубическими ячейками $Q_{k_i}$ так, чтобы $\lambda(P_k) \< \sum\limits_{i = 1}^{N_k} \lambda(Q_{k_i}) \< \lambda(P_k) + \frac{\eps}{2^k}$, т.е. тогда $E \subset \bigcup\limits_{i = 1}^\infty Q_i$ и $\sum\limits_{i = 1}^\infty \lambda(Q_i)= \sum\limits_{k = 1}^{\infty} \sum\limits_{i = 1}^{N_k} \lambda(Q_{k_i}) \< \sum\limits_{k = 1}^{\infty} \lambda(P_k) + \frac{\eps}{2^k} = \eps + \sum\limits_{k = 1}^{\infty} \lambda(P_k) < 2 \eps$
	Для шаров: возьмём покрытие $E$ кубическими ячейками $Q_i$ такими, что $\sum\limits_{i = 1}^\infty \lambda(Q_i) \< \frac{\eps}{m^{\sfrac m2}}$, тогда шары $B_i$, описанные вокруг этих ячеек (с радиусом $r = \frac{s\sqrt m}{2}$, $s$ --- сторона ячейки) будут тоже покрывать $E$ и $\lambda (B_i) \< \lambda(Q^*_i)$, где $Q^*_i$ --- кубическая ячейка, описанная вокруг шара (т.е. со стороной $2r$), но $\frac{\lambda(Q^*_i)}{\lambda(Q_i)} = \frac{(2r)^m}{\left(\frac{2r}{\sqrt m}\right)^m} = m^{\sfrac m2}$, значит $\sum\limits_{i = 1}^\infty \lambda(B_i) \< \sum\limits_{i = 1}^\infty \lambda(Q^*_i) = m^{\sfrac m2} \cdot \sum\limits_{i = 1}^\infty \lambda (Q_i) = m^{\sfrac m2} \cdot \frac{\eps}{m^{\sfrac m2}} = \eps $
	\end{enumerate} 
\end{prf}

\begin{slv}[https://www.youtube.com/live/FFhHi8qwuDM?si=jj9Xsn-sQMz_wZkE&t=7236]
	Все открытые и замкнутые множества измеримы
\end{slv}

\begin{zam}[https://www.youtube.com/live/FFhHi8qwuDM?si=bOO8QclsoFrTMLhZ&t=9029] \textbf{Канторовское множество:} \uns{\small(пример множества меры 0 мощности континуум)} пусть\\
	\begin{picture}(100,100)(-180,10)
		\put(0,94){\llap{$K_0 = [0, 1]$ \hspace{117pt} }}
		\put(0,97){\line(1,0){99}}
		\put(0,94){\line(0,1){6}}  \put(3.5,82){\small\llap0}
		\put(33,95){\line(0,1){4}} \put(36.5,82){\llap{$\frac13$}}
		\put(66,95){\line(0,1){4}} \put(69.5,82){\llap{$\frac23$}}
		\put(99,94){\line(0,1){6}} \put(102.5,82){\small\llap1}
		
		
		\put(0,60){\llap{$K_1 = [0, \frac13] \cup [\frac23, 1]$ \hspace{79pt} }}
		\put(0,63){\line(1,0){33}} 
		\put(0,60){\line(0,1){6}}  \put(3.5,48){\small\llap0}
		\put(11,61){\line(0,1){4}} \put(14.5,48){\footnotesize\llap{$\frac19$}}
		\put(22,61){\line(0,1){4}} \put(25.5,48){\footnotesize\llap{$\frac29$}}
		\put(33,60){\line(0,1){6}} \put(36.5,48){\llap{$\frac13$}}
		
		\put(66,63){\line(1,0){33}}
		\put(66,60){\line(0,1){6}} \put(69.5,48){\llap{$\frac23$}}
		\put(77,61){\line(0,1){4}} \put(80.5,48){\footnotesize\llap{$\frac79$}}
		\put(88,61){\line(0,1){4}} \put(91.5,48){\footnotesize\llap{$\frac89$}}
		\put(99,60){\line(0,1){6}} \put(102.5,48){\small\llap1}
		
		
		\put(0,30){\llap{$K_2 = [0, \frac19] \cup [\frac29, \frac13] \cup [\frac23, \frac79] \cup [\frac89, 1]$ \hspace{6pt}}}
		\put(0,33){\line(1,0){11}}
		\put(0,30){\line(0,1){6}} \put(3.5,18){\small\llap0}
		\put(11,30){\line(0,1){6}} \put(14.5,18){\footnotesize\llap{$\frac19$}}
		
		\put(22,33){\line(1,0){11}}
		\put(22,30){\line(0,1){6}} \put(25.5,18){\footnotesize\llap{$\frac29$}}
		\put(33,30){\line(0,1){6}} \put(36.5,18){\llap{$\frac13$}}
		
		\put(66,33){\line(1,0){11}}
		\put(66,30){\line(0,1){6}} \put(69.5,18){\llap{$\frac23$}}
		\put(77,30){\line(0,1){6}} \put(80.5,18){\footnotesize\llap{$\frac79$}}
		
		\put(88,33){\line(1,0){11}}
		\put(88,30){\line(0,1){6}} \put(91.5,18){\footnotesize\llap{$\frac89$}}
		\put(99,30){\line(0,1){6}} \put(102.5,18){\small\llap1}
	\end{picture}
	\hspace{-103pt}\raisebox{7pt}{\dots}\\
	Тогда канторовским множеством называется $K = \bigcap\limits_{n = 1}^{\infty}K_n$. Мощность $K$ --- континуум, т.к. каждому числу из канторовского множества $a \in K$ соответствует последовательность из 0 и 1 $(x_1, x_2, \ldots)$. Если $a \in K_n$ принадлежит отрезку $[\alpha, \alpha + \frac{1}{3^n}] \subset K_n$ \uns{(левому)}, то $x_n = 0$, а если $a$ принадлежит отрезку $[\beta - \frac{1}{3^n}, \beta] \subset K_n$ \uns{(правому)}, то $x_n = 0$, где $[\alpha, \beta] \subset K_{n-1}$ --- отрезок, которому принадлежит $a \in K_{n - 1}$. Это соответствие взаимно однозначно, и множество всех таких последовательностей имеет мощность континуум. Канторовское множество имеет меру 0, потому что $\A n\ \lambda(K) \< \lambda(K_n)$ и $\lambda(K_n) = 2^n \cdot \frac{1}{3^n} = \left(\frac{2}{3}\right)^n \xrightarrow[n \to 0]{} 0$
\end{zam}

\begin{zam}[https://youtu.be/w7W5PsIKw7A?si=kEybDZ-t5KG0wBwN&t=1186]
	\textbf{Пример неизмеримого по Лебегу множества:} пусть $\A a, b \in \rr$ $a \sim b$, если $a - b \in \mathbb{Q}$ (это отношение эквивалентности). Из каждого класса эквивалентности возьмём по одной точки и получим множество $A$. Можно считать, что $A \subset [0, 1]$. Рассмотрим множество \[B = \bigsqcup_{q \in \mathbb{Q} \cap [-1, 1]}(A + q)\]
	Объединение дизъюнктное, потому что если$\E x \in (A + q_1) \cap (A + q_2)$, то$\E a, b \in A : x = a+ q_1 = b + q_2$, но тогда $a - b = q_2 - q_1$, т.е. $a \sim b$ (но такого не может быть, т.к. в $A$ все точки из разных классов эквивалентности). Заметим, что $[0, 1] \subset B$ (потому что $\A x \in [0, 1] \E a \in A : x - a \in \mathbb Q$ \uns{(такое $a$ берётся из класса эквивалентности $x$)}, то есть $x = a + q \in B$). Также $B \subset [-1, 2]$. Тогда, если $A$ измеримо, то $\lambda(B) = \sum\limits_{\text{счётн.}} \lambda(A)$ (мера множества не меняется при его сдвиге --- следствие \ref{инв.отн.сдв.}). Так как $\lambda(B) \< \lambda[-1, 2] = 3$, то $\lambda(A) = 0$ (иначе бесконечная сумма одинаковых чисел $\lambda(A)$ будет равна бесконечности), т.е. $\lambda(B) = 0$. Но также $\lambda(B) \biger \lambda[0,1] = 1$. Значит $A$ --- неизмеримое множество.
\end{zam}