\subsection*{\S\ Измеримые функции}

\begin{opr}
	Пусть $X$ --- множество, $\mathcal A \subset 2^X$ --- $\sigma$-алгебра, $\mu \colon \mathcal A \to \overline\rr$ --- мера. Тогда тройка $(X, \mathcal A, \mu)$ называется \ybox{пространством с мерой}.
\end{opr}

\begin{opr}
	Пусть $X$ --- множество, тогда совокупность подмножеств $E_1, E_2, \ldots, E_n \subset X$ называется \urlybox{https://www.youtube.com/live/D_Nn53jVQxE?si=uGolwwyt-HPwiaQV&t=7815}{разбиением} множества $X$, если $X = \bigsqcup\limits_{i = 1}^n E_i$
\end{opr}

\begin{opr}\label{ступ.ф.}
	Функция $f \colon X \to \rr$ \uns{($X$ --- множество)} называется \urlybox{https://www.youtube.com/live/D_Nn53jVQxE?si=sp7YSdP4u3r9AvCO&t=7850}{ступенчатой}, если\E\ разбиение $E_1, E_2, \ldots, E_n$ множества $X$ такое, что \uns{$\A k \in \{\,1, 2, \ldots n \,\}$} \raisebox{0pt}[0pt][0pt]{$f \su{E_k} = c_k$}, где $c_k \in \rr$. Это разбиение называется \ybox{допустимым}.
\end{opr}

\begin{zam}[https://www.youtube.com/live/D_Nn53jVQxE?si=TLomW6CsIG2IkY16&t=7983]\begin{enumerate}\makeatletter\renewcommand{\p@enumi}{\thezam.}\makeatother
	\item\label{характ.функц.} \textit{Пример ступенчатой функции:} пусть $X$ --- множество, $E \subset X$, тогда \ybox{характеристическая} \ybox{функция} множества $E$ $\chi_E  \colon X \to \{\,0, 1\,\}$, где $\chi_E(x) = 1$, если $x \in E$ и $\chi_E(x) = 0$, если $x \notin E$ является ступенчатой \uns{(разбиение: $E, X \setminus E$)}
	
	\item Общий вид ступенчатой функции: \uns{\hypersetup{  linkcolor=mygray}(обозначения из \ref{ступ.ф.})\hypersetup{ linkcolor=linkcolor}} $f(x) = \sum\limits_{k = 1}^n \chi_{E_k} (x) \cdot c_k$ \uns{($x$ принадлежит какому-то одному $E_k$, поэтому все слагаемые будут равны 0, кроме одного, которое соответствует этому $E_k$, и на нём значение $f$ равно $c_k$)}
	
	\item\label{общ.разб.} Если $f, g \colon X \to \rr$ --- две ступенчатые функции \uns{(X --- множество)}, то\E\ \uns{(общее для них)} разбиение $E_1, E_2, \ldots E_n$ множества $X$ такое, что \uns{$\A k \in \{\,1, 2, \ldots n \,\}$} $f \su{E_k} =  c_k,\ g\su{E_k} = h_k$ \uns{($c_k, h_k \in \rr$)}\linebreak \textit{Доказательство:} из \ref{ступ.ф.}\E\ разбиения $A_1, A_2, \ldots A_s$ и $B_1, B_2, \ldots B_l$ множества $X$ такие, что \uns{$\A i \in \{\,1, 2, \ldots s \,\}$, $\A j \in \{\,1, 2, \ldots l \,\}$} $f \su{A_i} = c_i,\ g\su{B_j} = h_j$, значит $E_k = A_i \cap B_j$ 
	
	\item\label{произв.ступ.} Если $f, g \colon X \to \rr$ --- две ступенчатые функции \uns{(X --- множество)}, $\alpha \in \rr$, то функции $f + g,\ fg,\ \alpha f,\ \max(f, g),\ |f|$ тоже ступенчатые \uns{(общее разбиение для $f$ и $g$ (из предыдущего пункта) будет нужным разбиением для всех этих функций)}
\end{enumerate}\end{zam}

\begin{opr}\label{лебег.мн.}
	\!Пусть $f \colon E \subset X \to \overline \rr$ \uns{\small($X$ --- множество)}, $a \in \rr$.\! \urlybox{https://www.youtube.com/live/D_Nn53jVQxE?si=4_HSB1mD_eyG1Mjm&t=8369}{Лебеговыми множествами}\! функции $f$ называются множества \!$\{\, x \in E \mid f(x) < a\,\}$ \!(и аналогично с $\<,\ >,\ \biger$).\! Обозначение: $E(f < a)$
\end{opr}

\begin{zam}[https://www.youtube.com/live/D_Nn53jVQxE?si=SY2s0GtN3PMQn9R_&t=8496]\begin{enumerate}\makeatletter\renewcommand{\p@enumi}{\thezam.}\makeatother
	\item\label{1.лёгк.} $E(f > a) = E \setminus E(f \< a)$ \uns{(так как $\{\, x \in E \mid f(x) > a\,\} = \{\, x \in E \mid f(x) \< a \text{ --- не верно}\,\}$)}
	
	\item\label{2.лёгк.} $E(f \< a) = \bigcap\limits_{n = 1}^\infty E(f < a + \frac 1n)$
\end{enumerate}\end{zam}
\vspace{-5pt}
\begin{opr}\label{изм.ф.}
	Пусть $(X, \mathcal A, \mu)$ --- пространство с мерой, $f \colon X \to \overline \rr$, $E \in \mathcal{A}$. Функция $f$ называется \urlybox{https://www.youtube.com/live/D_Nn53jVQxE?si=NhdQRgAbbDncOgsB&t=8638}{измеримой на множестве} $E$, если $\A a \in \rr\ E(f < a) \in \mathcal A$. Функция $f$ называется \urlybox{https://www.youtube.com/live/D_Nn53jVQxE?si=NhdQRgAbbDncOgsB&t=8638}{измеримой}, если она измерима на множестве $X$. Если $X = \rmm$, $\mathcal{A} = \m$, $\mu$ --- мера Лебега, то измеримая функция называется \urlybox{https://www.youtube.com/live/D_Nn53jVQxE?si=NhdQRgAbbDncOgsB&t=8638}{измеримой по Лебегу}.
\end{opr}

\begin{zam}[https://www.youtube.com/live/D_Nn53jVQxE?si=HKSNoYsQLxpORzYv&t=8885]
	Пусть $(X, \mathcal A, \mu)$ --- пространство с мерой, $f \colon X \to \overline \rr$, $E \in \mathcal{A}$, тогда эквивалентно:
	\begin{multicols}{2}
		\begin{enumerate}
			\item $\A a \in \rr\ E(f < a) \in \mathcal A$
			
			\item $\A a \in \rr\ E(f \< a) \in \mathcal A$
			
			\item $\A a \in \rr\ E(f > a) \in \mathcal A$
			
			\item $\A a \in \rr\ E(f \biger a) \in \mathcal A$
		\end{enumerate}
	\end{multicols}
	\textit{Доказательство:} \rpr{1}{2}, \rpr{3}{4} Из замечания \ref{2.лёгк.} \uns{($\mathcal{A}$ замкнута относительно счётных пересечений)}. 
	\rpr{2}{3}, \rpr{4}{1} Из замечания \ref{1.лёгк.} \uns{(разность двух множеств из $\mathcal{A}$ принадлежит $\mathcal{A}$)}.
\end{zam}
\pagebreak
\begin{zam}[https://www.youtube.com/live/D_Nn53jVQxE?si=j9uDPlcRlLJ8BXNJ&t=9133] \textit{Пример измеримых функций:}
	\begin{enumerate}\makeatletter\renewcommand{\p@enumi}{зам.~\thezam.}\makeatother
	\item Пусть $(X, \mathcal A, \mu)$ --- пространство с мерой, $f \colon X \to \overline \rr$, $E \in \mathcal{A}$. Характеристическая функция множества $E$ (зам. \ref{характ.функц.}) измерима на $E$ \uns{(т.к. $\A a \in \rr\ E(\chi_E < a) = X \in \mathcal A$, если $a > 1$, $E(\chi_E < a) = E\sct \in \mathcal A$, если $0 < a \< 1$ и $E(\chi_E < a) = \no \in \mathcal A$, если $a \< 0$)}
	
	\item\label{непр.ф.изм.} Если $f \colon \rmm \to \rr$ непрерывна в \rmm, то $f$ измерима по Лебегу \uns{(т.к. при непрерывном отображении прообраз открытого множества открыт, и $\A a \in \rr\ \ \rmm(f < a)$ --- прообраз открытого множества $\{\, y \in \rr \mid y < a\,\}$, а открытые множества измеримы)}
	\end{enumerate}
\end{zam}

\begin{zam}[https://www.youtube.com/live/D_Nn53jVQxE?si=X-z7DBB2PacR-e1V&t=9380]
	\textit{Свойства измеримых функций:} пусть $(X, \mathcal A, \mu)$ --- пространство с мерой, $f \colon X \to \overline \rr$, $E \in \mathcal{A}$, тогда
	\begin{enumerate}\makeatletter\renewcommand{\p@enumi}{зам.~\thezam.}\makeatother
		\item Если $f$ измерима на $E$, то $\A a \in \rr\ E(f = a) \in \mathcal A$ \uns{\small(т.к. $E(f = a) = E(f \< a) \cap E(f \biger a) \in \mathcal{A}$)}
		
		\item Если $f$ измерима на $E$, то $-f$ и $\alpha f$ ($\A \alpha > 0$) измеримы на $E$ \uns{\small(т.к. $\A a \in \rr\ E(-f > a) = \{\, x \in E \mid -f(x) > a\,\} = \{\, x \in E \mid f(x) < -a\,\} = E(f < -a) \in \mathcal A$ и $E(\alpha f > a) = E(f > \frac a\alpha)$)}
		
		\item Если $f$ измерима на $E_k \in \mathcal A$ \uns{\small($k = 1, 2, \ldots$)}, то $f$ измерима на $\bigcup\limits_{k = 1}^\infty E_k$ \uns{\small(т.к.  $\A a \in \rr\ \left(\bigcup\limits_{k = 1}^\infty E_k\right)(f < a)\linebreak = \{\, x \in \raisebox{0pt}[0pt]{$\bigcup\limits_{k = 1}^\infty$} E_k \mid f(x) < a\,\} = \raisebox{0pt}[0pt]{$\bigcup\limits_{k = 1}^\infty$} \{\, x \in E_k \mid f(x) < a\,\} = \raisebox{0pt}[0pt]{$\bigcup\limits_{k = 1}^\infty$} \bigl(E_k(f < a)\bigr) \in \mathcal A$)} 
		
		\item\label{ф.изм.на подмнож.} Если $f$ измерима на $E$, $E'\!\subset E, E'\! \in \mathcal{A}$, то $f$ измерима на $E'$ \uns{\small(т.к. $E'(f < a) = E(f < a) \cap E') \in \mathcal{A}$)}
		
		\item Если $f$ измерима на $E$ и $f \ne 0$ на $E$, то $\frac 1f$ измерима на $E$ \uns{\small(т.к. для $a > 0$ $E(\frac 1f < a) = \bigl(E(f > \frac 1a) \cap E(f > 0)\bigr) \cup \bigl(E(f < \frac 1a) \cap E(f < 0)\bigr) \in \mathcal A$) --- т.е. разные множества при $f > 0$ и $f < 0$; аналогично для $a < 0, a = 0$}
		
		\item Если $f$ измерима на $E$, $f \biger 0$ на $E$, то $\A \alpha > 0$ $f^\alpha$ измерима на $E$ \uns{\small(т.к. $\A a \in \rr\ E(f^\alpha < a) = E(f < a^\frac 1\alpha) \in \mathcal{A}$)}
	\end{enumerate}
\end{zam}

\uns{\palka{
	Определение верхнего и нижнего предела последовательности $x_n \in \rr$:\\
	$\varlimsup\limits_{n \to \infty} x_n = \lim\limits_{n \to \infty} \sup\{\, x_n, x_{n + 1}, \ldots\,\}$\hspace{1.5cm}$\varliminf\limits_{n \to \infty} x_n = \lim\limits_{n \to \infty} \inf \{\, x_n, x_{n + 1}, \ldots\,\}$ 
}}

\begin{teor}[https://www.youtube.com/live/D_Nn53jVQxE?si=W2eSV2WkULR551h2&t=10108]{об измеримости пределов и супремумов}
	Пусть $(X, \mathcal A, \mu)$ --- пространство с мерой, $E \in \mathcal{A}$, функции $f_n \colon X \to \rr$ измеримы на $E$, тогда
	\begin{enumerate} \makeatletter\renewcommand{\p@enumi}{\theteor.}\makeatother
	\item $\sup\limits_n f_n(x)$, $\inf\limits_n f_n(x)$ измеримы на $E$
		
	\item $\varlimsup\limits_{n \to \infty} f_n(x)$, $\varliminf\limits_{n \to \infty} f_n(x)$ измеримы на $E$
	
	\item\label{изм.пред.} Если$\E \lim\limits_{n \to \infty} f_n(x)$, то он измерим на $E$
	\end{enumerate} 
\end{teor}

\begin{prf}
	\begin{enumerate}
		\item $E\bigl(\sup\limits_n f_n(x) > a\bigr) = \{\, x \in E \mid \sup\limits_n f_n(x) > a \,\} = \{\,x \in E \mid \!\!\!\E N : a < f_N(x) \uns{{}\< \sup\limits_n f_n(x)}\,\} = \bigcup\limits_{n = 1}^\infty \{\, x \in E \mid f_n(x) > a \,\} = \bigcup\limits_{n = 1}^\infty E (f_n > a) \in \mathcal A$. Аналогично с inf.
		
		\item $\varlimsup\limits_{n \to \infty} x_n = \inf\limits_n \sup \{\, f_n(x), f_{n + 1}(x), \ldots\,\}$. По первому пункту эта функция измерима на $E$.
		
		\item Если предел существует, то он совпадает с верхним пределом, который измерим на $E$ по второму пункту.
	\end{enumerate}
\end{prf}

\uns{\palka{%
	Определение положительной и отрицательной срезки для функции $f \colon X \to \rr$ ($X$ --- множество):\\
	\hspace*{3cm}$f_+(x) = \max\{\,f(x), 0\,\}\hspace{4cm} f_-(x) = \max\{\,-f(x), 0\,\}$
}}

\begin{slv}[https://www.youtube.com/live/D_Nn53jVQxE?si=-ruvK9hIfLccwsL8&t=10730]\label{изм.ср.}
	Пусть $(X, \mathcal A, \mu)$ --- пространство с мерой, $f \colon X \to \rr$ измерима на $E \in \mathcal{A}$, тогда $f_-, f_+, |f|$ измеримы на $E$ \uns{(т.к. $f_+(x) = \max\{\,f(x), 0\,\}, f_-(x) = \max\{\,-f(x), 0\,\}, |f(x)| = \max\{\,f(x), -f(x)\,\}$), т.е. по первому пункту теоремы они измеримы на $E$, потому что функции 0 и $f$ измеримы на $E$}
\end{slv}

\begin{teor}[https://www.youtube.com/live/D_Nn53jVQxE?si=JtFCmDrkbkpkwKzS&t=10835]{характеризация измеримых функций с помощью ступенчатых}\label{изм.ф.ступ.}%
	Пусть $(X, \mathcal A, \mu)$ --- пространство с мерой, $f \colon X \to \overline \rr$ --- неотрицательная, измеримая функция, тогда\E\ последовательность ступенчатых функций $f_n$ такая, что $\A x \in X$ $\uns{0 \<{}} f_n(x) \< f_{n + 1}(x) \< \ldots \< f(x)$ и $\lim\limits_{n \to \infty} f_n(x) = f(x)$
\end{teor}

\begin{prf}
	$\A n \in \{\,1, 2, \ldots \,\}$ построим функцию $g_n$. Для этого разобьём отрезок $[0, n]$ на $n^2$ равных частей длины $\frac 1n$ каждая. Обозначим $\A k \in \{\,1, 2, \ldots, n^2 - 1\,\}$ $E_k = X(\frac kn \< f < \frac{k + 1}{n}) \uns{{}\subset X}$ и $E_{n^2} = X(f \biger n)$, тогда $\{E_k\}_{k = 1}^{n^2}$ это разбиение множества $X$. Теперь $\A x \in X$ определим $g_n(x) = \sum\limits_{k = 0}^{n^2} \frac kn \chi_{E_k}(x)$, тогда $0 \< g_n(x) \< f(x)$ \uns{(т.к. $g_n(x) = \frac{k_0}{n} \biger 0$, где $k_0 \in \{\,1, 2, \ldots, n^2\,\} : x \in E_{k_0}$, и по определению множества $E_{k_0}$ на нём $\frac{k_0}{n} \< f$)}. Проверим, что $\A x \in X$ $\lim\limits_{n \to \infty}g_n(x) = f(x)$. Если $f(x)$ --- конечное число, то при $n > f(x)\ |f(x) - g_n(x)| < \frac 1n \xrightarrow[n \to \infty]{} 0$ \uns{(т.к. $x$ принадлежит некоторому $E_{k_0}$ и $k_0 \ne n^2$ (потому что $f(x) < n$), то по определению множества $E_{k_0}$ на нём $f < \frac{k_0 + 1}{n}$, значит $f(x) - g_n(x) < \frac{k_0 + 1}{n} - \frac{k_0}{n} = \frac 1n$)}, а если $f(x)$ --- бесконечность \uns{(т.е. $x \in E_{n^2}$)}, то $g_n(x) = n \xrightarrow[n \to \infty]{} \infty = f(x)$. В качестве $f_n(x)$ \uns{($\A n \in \{1, 2, \ldots\}$)} можно взять функцию $\max\{\, g_1(x), g_2(x), \ldots, g_n(x)\,\}$. Тогда  очевидно, выполняется свойство $\A x \in X$ $f_n(x) \< f_{n + 1}(x) \< \ldots \< f(x)$ и $f_n(x) \xrightarrow[n \to \infty]{} f(x)$, т.к. $g_n(x) \< f_n(x) \< f(x)$.
\end{prf}

\begin{zam}\label{зам. про изм. ступ. функц.}
	Из доказательства ясно, что все эти ступенчатые функции $f_n$ измеримы, так как допустимые разбиения для этих функций состоят из таких множеств $E_k$, что $E_k \in \mathcal A$. \uns{(в доказательстве $E_k = X(\frac kn \< f < \frac{k + 1}{n})$ и это принадлежит $\mathcal A$, т.к. $f$ --- измеримая функцию)}. То, что любая ступенчатая функция, у которой допустимое разбиение состоит из измеримых множеств, измерима, получается просто из определения измеримой функции, потому что любое множество Лебега для такой ступенчатой функции будет состоять из объединения этих измеримых множеств.
\end{zam}

\begin{slv}[https://www.youtube.com/live/D_Nn53jVQxE?si=BWWH7fyNDeDPB5yH&t=11690]
	Пусть $f \colon X \to \rr$ --- измерима \uns{($(X, \mathcal A, \mu)$ --- пространство с мерой)}, тогда\E\ ступенчатые функции $f_n \colon X \to \rr$ такие, что $\A x \in X\ \lim\limits_{n \to \infty} f_n(x) = f(x)$ и $\A n\ |f_n(x)| \< |f(x)|$ \textit{Доказательство:} $f$ можно разложить в разность положительных измеримых функций $f = f_+ - f_-$ (сл. \ref{изм.ср.}) и взять $f_n = (f_+)_n - (f_-)_n$, где $(f_+)_n$ и $(f_-)_n$ --- соответствующие последовательности ступенчатых функций, полученные из теоремы  \ref{изм.ф.ступ.}. В этой разности, при $x$, в которых $f$ отрицательна, слагаемое $(f_+)_n(x)$ нулевое, а когда $f$ положительна, то второе слагаемое 0)
\end{slv}

\begin{slv}[https://www.youtube.com/live/3LvgYceATKY?si=f2_NDoyMYDquQWRc&t=7926]
	Пусть $f, g \colon X \to \rr$ --- измеримы \uns{($(X, \mathcal A, \mu)$ --- пространство с мерой)}, тогда функция $f \cdot g$ измерима (и считаем, что $0 \cdot \infty = 0$). \textit{Доказательство:} из предыдущего следствия\E\ ступенчатые функции $f_n, g_n \colon X \to \rr$ такие, что $\A x \in X\ \lim\limits_{n \to \infty}f_n(x) = f(x)$ и $\lim\limits_{n \to \infty}g_n(x) = g(x)$), значит $\lim\limits_{n \to \infty}f_n(x)g_n(x) = f(x)g(x)$. Произведение ступенчатых функций --- ступенчатая функция (зам.~\ref{произв.ступ.}), поэтому получилось, что $fg$ это предел последовательности ступенчатых функций, которые измеримы (\ref{зам. про изм. ступ. функц.}). Значит $fg$ --- измерима, как предел измеримых функций (т. \ref{изм.пред.}).
\end{slv}

\begin{slv}[https://www.youtube.com/live/3LvgYceATKY?si=7WMGceaqVhikXnEY&t=8177]
	Пусть $f, g \colon X \to \rr$ --- измеримы \uns{($(X, \mathcal A, \mu)$ --- пространство с мерой)}, тогда функция $f + g$ измерима (и считаем, что $\A x \in X \{\,f(x), g(x)\,\} \ne \{\,+\infty, -\infty\,\}$). Доказывается аналогично предыдущему следствию.
\end{slv}
\pagebreak
\begin{teor}[https://www.youtube.com/live/3LvgYceATKY?si=ZYlgdvhg-6AVMtL_&t=8408]{измеримость функции непрерывной на множестве полной меры}
	Пусть функция $f \colon E \subset \rmm \to \overline{\rr}$ непрерывна на $E' = E \setminus e$, где $e \subset E$ и $\lambda(e) = 0$ \uns{($E,e$ --- множества измеримые по Лебегу)}. Тогда функция $f$ измерима на $E$.
\end{teor}

\begin{prf}
	Нужно доказать, что $\A a \in \rr$ множество $E(f < a)$ --- измеримо (\ref{лебег.мн.}, \ref{изм.ф.}). Оно измеримо, т.к. $E(f < a) = E'(f < a) \cup e(f < a)$. Первое из этих множеств измеримо, потому что $f$ непрерывна на $E'$ по условию (\ref{непр.ф.изм.}), второе измеримо, потому что оно содержится в $e$ и $\lambda(e) = 0$ \uns{\hypersetup{  linkcolor=mygray}(мера Лебега является полной --- \ref{полная мера}, \ref{мера лебега})\hypersetup{ linkcolor=linkcolor}}.
\end{prf}

\begin{slv}[https://www.youtube.com/live/3LvgYceATKY?si=gOwssnHN2eKqCbab&t=9162]\label{изм. на e}
	Если $(X, \mathcal A, \mu)$ --- пространство с мерой, $E, e, \in \mathcal A : e \subset E, \mu(e) = 0$, и функция $f \colon E \to \overline{\rr}$ измерима на $E' = E \setminus e$, то $f$ можно изменить на $e$ так, чтобы она стала измеримой (например, определить $f \su{e} = 0$, и доказать измеримость новой $f$ аналогично тому, как в теореме)
\end{slv}

\begin{slv}
	Монотонная функция измерима \dots 
\end{slv}
